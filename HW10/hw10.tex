\documentclass{article}
\usepackage{amsmath}
\usepackage{zed-csp, graphicx}
\input{handout}

\begin{document}

\homework{}{05 November 2018}{Homework \#10: Concurrency}{}

\begin{enumerate}

\item Traces and Specifications:

\begin{enumerate}
\item Enumerate the traces for the following process \verb"P":
\begin{verbatim}
 P = (a -> a -> END | b -> a -> END).
\end{verbatim}
$traces == $
$ \{ \langle a, a \rangle, \langle b, a \rangle, \langle a \rangle, \langle b \rangle \} $ \\

\item Enumerate the traces for the following process \verb"P":
\begin{verbatim}
 P1 = (a -> a -> END).
 P2 = (a -> b -> END | a -> c -> END).
 ||P = (P1 || P2).
\end{verbatim}
$traces ==  \{ \langle a, b \rangle, \langle a,c \rangle \} $ \\
\\
It is important to mention that P1 seems to always deadlock as it requires 2 $a$ actions to reach the $END$ but P2 can only go through 1 a per run. \\
\end{enumerate}

\item More Concurrency:

Consider the two processes \texttt{STUDENT} and \texttt{TEACHER}, where
\begin{align*}
    \alpha\ \mathtt{STUDENT} &= \{\mathtt{do\_hw}, \mathtt{hand\_in}, \mathtt{pass}, \mathtt{fail}, \mathtt{cheer}, \mathtt{curse}\} \\
    \alpha\ \mathtt{TEACHER} &= \{\mathtt{hand\_in}, \mathtt{grade}, \mathtt{pass}, \mathtt{fail}, \mathtt{grumble}\}
\end{align*}
The student repeatedly does her homework, hands it in, and gets a pass or fail---cheering when she
passes and cursing when she fails. The teacher repeatedly collects the homework, grades it, and then assigns a pass or
fail grade---grumbling after any time that he has to give out a failing grade.

\begin{enumerate}
\item Write an FSP process that characterizes the student and show a diagram that indicates its
  behavior. \\
\begin{verbatim}
STUDENT = (do_hw -> hand_in -> STUDENT_SEND),

STUDENT_SEND = (pass -> cheer -> STUDENT | fail -> curse -> STUDENT).  
\end{verbatim}

\begin{center}
\includegraphics[width=3in]{STUDENT.png}
\end{center}

\item Write an FSP process that characterizes the teacher and show a diagram that indicates its
  behavior. \\

\begin{verbatim}
TEACHER = (hand_in->grade->TEACHER_GRADE_RESULT),

TEACHER_GRADE_RESULT = (pass -> TEACHER | fail -> grumble -> TEACHER).
\end{verbatim}

\begin{center}
\includegraphics[width=3in]{TEACHER.png}
\end{center}  

\item Produce an LTS graph for \texttt{STUDENT || TEACHER}.

\item What happens to this process if we augment \texttt{STUDENT}'s alphabet with the \texttt{grumble} event and
have her grumble before doing her homework? Why does this occur?

\item If your answer to the previous question involves deadlock, list
two ways that you might change the definition to avoid this
unintended problem. ({\sc Note}: You may not change the order in which events happen. For example, do not move the student's \texttt{grumble} event after her \texttt{hand\_in} event. Preserve the intended behavior of the model.)
\end{enumerate}

\item Exercises Based on MK06

Consider the model of the client--server system described in section~3.1.4 of MK06.
\begin{enumerate}
\item Extend the model of the client--server system so that more than one client can use the server. Your model should support an arbitrary number of clients (\texttt{N}).

\item Modify your new model of the client--server system so that a client's call may terminate with a \texttt{timeout} action rather than a response from the server. (Do not modify the server process.) What condition results from this modification?
\end{enumerate}
\end{enumerate}

\end{document}
