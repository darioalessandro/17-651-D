%%%%%%%%%%%%%%%%%%%%%%%%%%%%%%%%%%%%%%%%%%%%%%%%%%%%%%%%%%%%%%%%
%  6.826 (POCS Seminar) macro file for handouts and problem sets.
%
% You should save this file as handout.tex
%
% Your main LaTeX file should look like this:
%
%        \documentstyle[12pt]{article}
%
%        %%%%%%%%%%%%%%%%%%%%%%%%%%%%%%%%%%%%%%%%%%%%%%%%%%%%%%%%%%%%%%%%
%  6.826 (POCS Seminar) macro file for handouts and problem sets.
%
% You should save this file as handout.tex
%
% Your main LaTeX file should look like this:
%
%        \documentstyle[12pt]{article}
%
%        %%%%%%%%%%%%%%%%%%%%%%%%%%%%%%%%%%%%%%%%%%%%%%%%%%%%%%%%%%%%%%%%
%  6.826 (POCS Seminar) macro file for handouts and problem sets.
%
% You should save this file as handout.tex
%
% Your main LaTeX file should look like this:
%
%        \documentstyle[12pt]{article}
%
%        %%%%%%%%%%%%%%%%%%%%%%%%%%%%%%%%%%%%%%%%%%%%%%%%%%%%%%%%%%%%%%%%
%  6.826 (POCS Seminar) macro file for handouts and problem sets.
%
% You should save this file as handout.tex
%
% Your main LaTeX file should look like this:
%
%        \documentstyle[12pt]{article}
%
%        \input{handout}
%%%%%%%%%%%%%%%%%%%%%%%%%%%%%%%%%%%%%%%%%%%%%%%%%%%%%%%%%%%%%%%%

\oddsidemargin 0in
\evensidemargin 0in
\marginparwidth 40pt
\marginparsep 10pt
\topmargin 0pt
\headsep 0in
\headheight 0in
\textheight 8.5in
\textwidth 6in
\brokenpenalty=10000

% \handout{number}{date}{title}

\newcommand{\handout}[3]{


\begin{center}
\rule{\textwidth}{.0075in} \\
\rule[3mm]{\textwidth}{.0075in}\\

CMU 17-651\hfill Models of Software Systems\hfill Fall 2018\\[3ex]

{\Large\bf #3}\\[3ex]

Dario A Lencina-Talarico \hfill {\bf Handout #1} \hfill #2

\rule{\textwidth}{.0075in} \\
\rule[3mm]{\textwidth}{.0075in} \\
\end{center}

}

% \homework{number}{date}{title}{due-date}
\newcommand{\homework}[4]{

\begin{center}
\rule{\textwidth}{.0075in} \\
\rule[3mm]{\textwidth}{.0075in}\\

CMU 17-651\hfill Models of Software Systems\hfill Fall 2018\\[3ex]

{\Large\bf #3} \\[3ex]

Dario A Lencina Talarico \hfill  #1  \hfill Due: #2\\

\rule{\textwidth}{.0075in} \\
\rule[3mm]{\textwidth}{.0075in} \\
\end{center}

%\noindent
%{\bf Due date: #4}

}

% \solutionset{number}{date}{title}{due-date}
\newcommand{\solutionset}[4]{

\begin{center}
\rule{\textwidth}{.0075in} \\
\rule[3mm]{\textwidth}{.0075in}\\

CMU 17-651\hfill Models of Software Systems\hfill Fall 2016\\[3ex]

{\Large\bf #3} \\[3ex]

Garlan  \hfill  Solutions for Homework #1  \hfill  #2\\

\rule{\textwidth}{.0075in} \\
\rule[3mm]{\textwidth}{.0075in} \\
\end{center}

%\noindent
%{\bf Due date: #4}

}

% \problem{problem-number}
\newcommand{\problem}[1]{
\vspace{2ex}
\noindent
{\bf Problem #1.}

}

% \solution{solution-number}{points}
\newcommand{\solution}[2]{
\vspace{3ex}
\noindent
{\bf Problem #1}  (#2 points)

}

\newcommand{\cscomment}{
\vspace{1ex}
\noindent Comments: }

% \parts{part-alphabet}{points}
\newcommand{\parts}[2]{
\vspace{2ex}
\noindent
{\bf (#1)}  (#2 points)

}

% \problems{problems-number}{points}
\newcommand{\problems}[2]{
\vspace{3ex}
\noindent
{\bf Problem #1}  (#2 points)

}

\newenvironment{symbolfootnotes}{\def\thefootnote{\fnsymbol{footnote}}}{}

%%%%%%%%%%%%%%%%%%%%%%%%%%%%%%%%%%%%%%%%%%%%%%%%%%%%%%%%%%%%%%%%

\oddsidemargin 0in
\evensidemargin 0in
\marginparwidth 40pt
\marginparsep 10pt
\topmargin 0pt
\headsep 0in
\headheight 0in
\textheight 8.5in
\textwidth 6in
\brokenpenalty=10000

% \handout{number}{date}{title}

\newcommand{\handout}[3]{


\begin{center}
\rule{\textwidth}{.0075in} \\
\rule[3mm]{\textwidth}{.0075in}\\

CMU 17-651\hfill Models of Software Systems\hfill Fall 2018\\[3ex]

{\Large\bf #3}\\[3ex]

Dario A Lencina-Talarico \hfill {\bf Handout #1} \hfill #2

\rule{\textwidth}{.0075in} \\
\rule[3mm]{\textwidth}{.0075in} \\
\end{center}

}

% \homework{number}{date}{title}{due-date}
\newcommand{\homework}[4]{

\begin{center}
\rule{\textwidth}{.0075in} \\
\rule[3mm]{\textwidth}{.0075in}\\

CMU 17-651\hfill Models of Software Systems\hfill Fall 2018\\[3ex]

{\Large\bf #3} \\[3ex]

Dario A Lencina Talarico \hfill  #1  \hfill Due: #2\\

\rule{\textwidth}{.0075in} \\
\rule[3mm]{\textwidth}{.0075in} \\
\end{center}

%\noindent
%{\bf Due date: #4}

}

% \solutionset{number}{date}{title}{due-date}
\newcommand{\solutionset}[4]{

\begin{center}
\rule{\textwidth}{.0075in} \\
\rule[3mm]{\textwidth}{.0075in}\\

CMU 17-651\hfill Models of Software Systems\hfill Fall 2016\\[3ex]

{\Large\bf #3} \\[3ex]

Garlan  \hfill  Solutions for Homework #1  \hfill  #2\\

\rule{\textwidth}{.0075in} \\
\rule[3mm]{\textwidth}{.0075in} \\
\end{center}

%\noindent
%{\bf Due date: #4}

}

% \problem{problem-number}
\newcommand{\problem}[1]{
\vspace{2ex}
\noindent
{\bf Problem #1.}

}

% \solution{solution-number}{points}
\newcommand{\solution}[2]{
\vspace{3ex}
\noindent
{\bf Problem #1}  (#2 points)

}

\newcommand{\cscomment}{
\vspace{1ex}
\noindent Comments: }

% \parts{part-alphabet}{points}
\newcommand{\parts}[2]{
\vspace{2ex}
\noindent
{\bf (#1)}  (#2 points)

}

% \problems{problems-number}{points}
\newcommand{\problems}[2]{
\vspace{3ex}
\noindent
{\bf Problem #1}  (#2 points)

}

\newenvironment{symbolfootnotes}{\def\thefootnote{\fnsymbol{footnote}}}{}

%%%%%%%%%%%%%%%%%%%%%%%%%%%%%%%%%%%%%%%%%%%%%%%%%%%%%%%%%%%%%%%%

\oddsidemargin 0in
\evensidemargin 0in
\marginparwidth 40pt
\marginparsep 10pt
\topmargin 0pt
\headsep 0in
\headheight 0in
\textheight 8.5in
\textwidth 6in
\brokenpenalty=10000

% \handout{number}{date}{title}

\newcommand{\handout}[3]{


\begin{center}
\rule{\textwidth}{.0075in} \\
\rule[3mm]{\textwidth}{.0075in}\\

CMU 17-651\hfill Models of Software Systems\hfill Fall 2018\\[3ex]

{\Large\bf #3}\\[3ex]

Dario A Lencina-Talarico \hfill {\bf Handout #1} \hfill #2

\rule{\textwidth}{.0075in} \\
\rule[3mm]{\textwidth}{.0075in} \\
\end{center}

}

% \homework{number}{date}{title}{due-date}
\newcommand{\homework}[4]{

\begin{center}
\rule{\textwidth}{.0075in} \\
\rule[3mm]{\textwidth}{.0075in}\\

CMU 17-651\hfill Models of Software Systems\hfill Fall 2018\\[3ex]

{\Large\bf #3} \\[3ex]

Dario A Lencina Talarico \hfill  #1  \hfill Due: #2\\

\rule{\textwidth}{.0075in} \\
\rule[3mm]{\textwidth}{.0075in} \\
\end{center}

%\noindent
%{\bf Due date: #4}

}

% \solutionset{number}{date}{title}{due-date}
\newcommand{\solutionset}[4]{

\begin{center}
\rule{\textwidth}{.0075in} \\
\rule[3mm]{\textwidth}{.0075in}\\

CMU 17-651\hfill Models of Software Systems\hfill Fall 2016\\[3ex]

{\Large\bf #3} \\[3ex]

Garlan  \hfill  Solutions for Homework #1  \hfill  #2\\

\rule{\textwidth}{.0075in} \\
\rule[3mm]{\textwidth}{.0075in} \\
\end{center}

%\noindent
%{\bf Due date: #4}

}

% \problem{problem-number}
\newcommand{\problem}[1]{
\vspace{2ex}
\noindent
{\bf Problem #1.}

}

% \solution{solution-number}{points}
\newcommand{\solution}[2]{
\vspace{3ex}
\noindent
{\bf Problem #1}  (#2 points)

}

\newcommand{\cscomment}{
\vspace{1ex}
\noindent Comments: }

% \parts{part-alphabet}{points}
\newcommand{\parts}[2]{
\vspace{2ex}
\noindent
{\bf (#1)}  (#2 points)

}

% \problems{problems-number}{points}
\newcommand{\problems}[2]{
\vspace{3ex}
\noindent
{\bf Problem #1}  (#2 points)

}

\newenvironment{symbolfootnotes}{\def\thefootnote{\fnsymbol{footnote}}}{}

%%%%%%%%%%%%%%%%%%%%%%%%%%%%%%%%%%%%%%%%%%%%%%%%%%%%%%%%%%%%%%%%

\oddsidemargin 0in
\evensidemargin 0in
\marginparwidth 40pt
\marginparsep 10pt
\topmargin 0pt
\headsep 0in
\headheight 0in
\textheight 8.5in
\textwidth 6in
\brokenpenalty=10000

% \handout{number}{date}{title}

\newcommand{\handout}[3]{


\begin{center}
\rule{\textwidth}{.0075in} \\
\rule[3mm]{\textwidth}{.0075in}\\

CMU 17-651\hfill Models of Software Systems\hfill Fall 2018\\[3ex]

{\Large\bf #3}\\[3ex]

Garlan \hfill {\bf Handout #1} \hfill #2

\rule{\textwidth}{.0075in} \\
\rule[3mm]{\textwidth}{.0075in} \\
\end{center}

}

% \homework{number}{date}{title}{due-date}
\newcommand{\homework}[4]{

\begin{center}
\rule{\textwidth}{.0075in} \\
\rule[3mm]{\textwidth}{.0075in}\\

CMU 17-651\hfill Models of Software Systems\hfill Fall 2018\\[3ex]

{\Large\bf #3} \\[3ex]

Dario A Lencina-Talarico \hfill  #1  \hfill Due: #2\\

\rule{\textwidth}{.0075in} \\
\rule[3mm]{\textwidth}{.0075in} \\
\end{center}

%\noindent
%{\bf Due date: #4}

}

% \solutionset{number}{date}{title}{due-date}
\newcommand{\solutionset}[4]{

\begin{center}
\rule{\textwidth}{.0075in} \\
\rule[3mm]{\textwidth}{.0075in}\\

CMU 17-651\hfill Models of Software Systems\hfill Fall 2018\\[3ex]

{\Large\bf #3} \\[3ex]

Dario A Lencina-Talarico  \hfill  Solutions for Homework #1  \hfill  #2\\

\rule{\textwidth}{.0075in} \\
\rule[3mm]{\textwidth}{.0075in} \\
\end{center}

%\noindent
%{\bf Due date: #4}

}

% \problem{problem-number}
\newcommand{\problem}[1]{
\vspace{2ex}
\noindent
{\bf Problem #1.}

}

% \solution{solution-number}{points}
\newcommand{\solution}[2]{
\vspace{3ex}
\noindent
{\bf Problem #1}  (#2 points)

}

\newcommand{\cscomment}{
\vspace{1ex}
\noindent Comments: }

% \parts{part-alphabet}{points}
\newcommand{\parts}[2]{
\vspace{2ex}
\noindent
{\bf (#1)}  (#2 points)

}

% \problems{problems-number}{points}
\newcommand{\problems}[2]{
\vspace{3ex}
\noindent
{\bf Problem #1}  (#2 points)

}

\newenvironment{symbolfootnotes}{\def\thefootnote{\fnsymbol{footnote}}}{}
