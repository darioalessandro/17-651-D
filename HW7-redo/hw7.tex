\documentclass{article}
\usepackage{zed-csp}
\usepackage{savesym}
\usepackage{amsfonts}
\input{handout}
\newenvironment{spec}{
 \vspace*{8pt}
 \begin{center}
 \begin{minipage}{5in}
 \renewcommand{\baselinestretch}{1}
 }{
 \end{minipage}
 \end{center}
 \vspace*{8pt}
}
\newcommand{\estimates}{\mathrel{\hat{=}}}
\begin{document}

\homework{}{15 October 2018}{Homework \#7: Invariants and Introduction to Z}{}

\noindent \textbf{Part 1: Invariants}\\

\noindent Consider the Diverging Counter example of Chapter 10 of GWC09. Prove that $x+y=0$ is an invariant of the $DivergingCounter$ state machine.\\
\\
DivergingCounter = $($ \\
$[ x, y : \mathbb{Z} ],$ \\
$\{ s: [x, y : \mathbb{Z}] | s(x) = -s(y) \},$ \\
$\{poke(i: \mathbb{Z})\},$ \\
$\delta$ == \\

\begin{spec}
\begin{tabbing}
\indent {\em poke}\={\em (i: $\mathbb{Z}$)}\\
           \> {\bf pre} $i > 0$\\
           \> {\bf post} $x' = x + i \wedge y' = y - i$
\end{tabbing}
\end{spec}

$)$.
\\
1. Base case: show that $\theta$ holds in the initial state. \\
Here there's only one initial state: \\
$[x = 0;y = 0 ]$ \\ 
Proof: \\
\begin{tabular}{l ll lll}
     x + y = 0 \\ &    &  \\
     = &   & [initial state] \\
     0 + 0 = 0 & & \\
     = &   & [aritmetic] \\
     0 = 0  \\
\end{tabular} \\
2. Induction step on inc: \\
\\
Show: $\theta (s), pre(s), post(s, s') \vdash \theta(s')$ \\
That is, from $x' = x + i \wedge y' = y - i$ \\
\\
$ \theta (s) == x + y = 0 $ \\
prove that $ x' + y' = 0 $ \\
Proof: \\
\begin{tabular}{l ll lll}
     y' \\ &    &  \\
     = &   & [postcondition] \\
     y - i & & \\
     = &   &  [hypothesis introduction  y + x = 0, or y = -x] \\
     -x - i \\
     = & & [factorizing] \\ 
     -(x + i) \\
     = & & [replacing definition of x'] \\
     -x' \\
     \\
    y' = -x' is equivalent to y' + x' = 0 \\
\end{tabular} \\
\noindent (\textsc{Note:} In your proof use style C (in Section 10.1.1) of reasoning about invariants and a similar degree of formalism as in the lecture on this topic.)

\vspace{12pt}

\noindent \textbf{Part 2: Z} \\

\noindent \textsc{Note}: For this part of the assignment you must format your
answers using \LaTeX and typecheck the answers using {\em fuzz},  Z-EVES, or the Community Z tools.  \\

\noindent Write a Z specification of the following system. Your specification should include sufficient explanatory
prose to make it easily understandable. (The prose is important---answers with little
or no prose will receive a low grade.)\\[2ex]
A teacher wants to keep a register of students in the class, and to record which students have completed their homework. \\[2ex]
Let the given set $Student$ represent the set of all students who might ever be enrolled in a class: 
\begin{zed}
[Student]
\end{zed}
Specify each of the following:
\begin{enumerate}
\item The state space for a register. \\
  \textbf{Reasoning:} 
  I propose using two powersets of students, one to represent the enrolled students and another one for the students that completed
  their homework. \\
  I added the invariant completed $\subseteq$ enrolled to capture the fact that all students that completed their homework have to be enrolled \\
\textsc{Hint}: use two sets of students:
\begin{schema}{Register}
enrolled: \power Student\\
completed: \power Student
\where
completed \subseteq enrolled \\
\end{schema}
\item An operation to enroll a new student. \\
  \textbf{Reasoning:}
  I propose using the union operation between the currently enrolled vehicles and the new student. \\
  It is important to mention that the union operation is defined for sets, that is why $student?$ is  wrapped in a set. \\
\begin{schema}{Enroll}
  \Delta Register \\
  student?:Student
  \where
  student? \notin enrolled \\
  enrolled' = enrolled \union $\{student?\}$ \\
  completed' = completed
\end{schema}  
\item The initial state(s)  for your state space. \\
  \\
  \textbf{Reasoning:} The proposed z-spec is to initialize both sets as empty sets. \\
\begin{schema}{InitRegister}
Register
\where
enrolled = \emptyset \\
completed = \emptyset
\end{schema}
\item An operation to record that a student (already enrolled in class) has completed the homework.
  \textbf{Reasoning:} The proposed Z spec adds student? to the completed power set by definining a union. \\
  \begin{schema}{RegisterHomeworkCompletion}
  \Delta Register \\
  student?:Student
  \where
  student? \in enrolled \\
  completed' = completed \union $\{student?\}$ \\
  enrolled' = enrolled
  \end{schema} 
\item An operation to inquire whether a student (who must be enrolled) has completed the homework (the answer is to be either `Yes' or `No'). \\
  \textbf{Reasoning:} The previous statement specifies the expected return values, to meet the requirements, we need to define a new type to answer: \\
  \\
  Answer ::= Yes $\mid$ No \\
  HasCompletedTheHomeworkSuccess is used to express the successful HasCompletedTheHomework check. it returns answer:Yes, all it does is to check if the student is in the completed set. \\
  \begin{schema}{HasCompletedTheHomeworkSuccess}
  \Xi Register \\
  student?:Student \\
  answer!:Answer 
  \where
  student? \in completed \\
  answer!: Yes
  \end{schema}
  HasCompletedTheHomeworkErr is used to express the error scenario, meaning, student is not in the completed set, it returns answer:No \\
  \begin{schema}{HasCompletedTheHomeworkErr}
  \Xi Register \\
  student?:Student \\
  answer!:Answer 
  \where
  student? \notin completed \\
  answer!: No
  \end{schema}
  \textbf{Reasoning:} Only one of the expressions will meet the preconditions so $\lor$ is the right relation to use: \\
  \\
  HasCompletedTheHomework $\estimates HasCompletedTheHomeworkSuccess \lor HasCompletedTheHomeworkErr $ \\
  
\item A robust version of the system. (Be sure to use the schema calculus, as illustrated by the class lecture and the paper by Spivey on Z.) \\
  Declaring type to return a result: \\
REPORT ::= ok $\mid$ already-enrolled $\mid$ not-enrolled \\
\textbf{Reasoning:} We propose using REPORT to return the success or failure of each operation. \\
We also need to define some helper schemas to embed the RESULT. \\
Success is used when the operation has been successful. \\
  \begin{schema}{Success}
    result!:REPORT
  \where
    result!: ok
  \end{schema}
  NotEnrolled is used to capture the scenario where the student is not enrolled in the register, all we do is checking if the student is in the enrolled set. \\
  \begin{schema}{NotEnrolled}
    \Xi Register \\
    student?:Student \\
    result!:REPORT
    \where
    student? \notin enrolled \\
    result!: $not-enrolled$
  \end{schema}
  AlreadyEnrolled is used to handle the error scenario of the enroll operation, all we do is a precondition that checks if the student is in the enrolled set. \\
  \begin{schema}{AlreadyEnrolled}
    \Xi Register \\
    student?:Student \\
    result!:REPORT
    \where
    student? \in enrolled \\
    result!: $already-enrolled$
  \end{schema}
  Robust versions of the operations: \\
  \\
  RREnroll $\estimates (Enroll \land Success) \lor AlreadyEnrolled$\\
  RRegisterHomeworkCompletion $\estimates (RegisterHomeworkCompletion \land Success) \lor NotEnrolled $  \\
  HasCompletedTheHomework is already a robust method so no need to modify. \\
  \\
\end{enumerate}

\end{document}
