\documentstyle[10pt,epsf,fuzz]{article}

%%%%%%%%%%%%%%%%%%%%%%%%%%%%%%%%%%%%%%%%%%%%%%%%%%%%%%%%%%%%%%%%
%  6.826 (POCS Seminar) macro file for handouts and problem sets.
%
% You should save this file as handout.tex
%
% Your main LaTeX file should look like this:
%
%        \documentstyle[12pt]{article}
%
%        %%%%%%%%%%%%%%%%%%%%%%%%%%%%%%%%%%%%%%%%%%%%%%%%%%%%%%%%%%%%%%%%
%  6.826 (POCS Seminar) macro file for handouts and problem sets.
%
% You should save this file as handout.tex
%
% Your main LaTeX file should look like this:
%
%        \documentstyle[12pt]{article}
%
%        %%%%%%%%%%%%%%%%%%%%%%%%%%%%%%%%%%%%%%%%%%%%%%%%%%%%%%%%%%%%%%%%
%  6.826 (POCS Seminar) macro file for handouts and problem sets.
%
% You should save this file as handout.tex
%
% Your main LaTeX file should look like this:
%
%        \documentstyle[12pt]{article}
%
%        \input{handout}
%%%%%%%%%%%%%%%%%%%%%%%%%%%%%%%%%%%%%%%%%%%%%%%%%%%%%%%%%%%%%%%%

\oddsidemargin 0in
\evensidemargin 0in
\marginparwidth 40pt
\marginparsep 10pt
\topmargin 0pt
\headsep 0in
\headheight 0in
\textheight 8.5in
\textwidth 6in
\brokenpenalty=10000

% \handout{number}{date}{title}

\newcommand{\handout}[3]{


\begin{center}
\rule{\textwidth}{.0075in} \\
\rule[3mm]{\textwidth}{.0075in}\\

CMU 17-651\hfill Models of Software Systems\hfill Fall 2018\\[3ex]

{\Large\bf #3}\\[3ex]

Dario A Lencina-Talarico \hfill {\bf Handout #1} \hfill #2

\rule{\textwidth}{.0075in} \\
\rule[3mm]{\textwidth}{.0075in} \\
\end{center}

}

% \homework{number}{date}{title}{due-date}
\newcommand{\homework}[4]{

\begin{center}
\rule{\textwidth}{.0075in} \\
\rule[3mm]{\textwidth}{.0075in}\\

CMU 17-651\hfill Models of Software Systems\hfill Fall 2018\\[3ex]

{\Large\bf #3} \\[3ex]

Dario A Lencina Talarico \hfill  #1  \hfill Due: #2\\

\rule{\textwidth}{.0075in} \\
\rule[3mm]{\textwidth}{.0075in} \\
\end{center}

%\noindent
%{\bf Due date: #4}

}

% \solutionset{number}{date}{title}{due-date}
\newcommand{\solutionset}[4]{

\begin{center}
\rule{\textwidth}{.0075in} \\
\rule[3mm]{\textwidth}{.0075in}\\

CMU 17-651\hfill Models of Software Systems\hfill Fall 2016\\[3ex]

{\Large\bf #3} \\[3ex]

Garlan  \hfill  Solutions for Homework #1  \hfill  #2\\

\rule{\textwidth}{.0075in} \\
\rule[3mm]{\textwidth}{.0075in} \\
\end{center}

%\noindent
%{\bf Due date: #4}

}

% \problem{problem-number}
\newcommand{\problem}[1]{
\vspace{2ex}
\noindent
{\bf Problem #1.}

}

% \solution{solution-number}{points}
\newcommand{\solution}[2]{
\vspace{3ex}
\noindent
{\bf Problem #1}  (#2 points)

}

\newcommand{\cscomment}{
\vspace{1ex}
\noindent Comments: }

% \parts{part-alphabet}{points}
\newcommand{\parts}[2]{
\vspace{2ex}
\noindent
{\bf (#1)}  (#2 points)

}

% \problems{problems-number}{points}
\newcommand{\problems}[2]{
\vspace{3ex}
\noindent
{\bf Problem #1}  (#2 points)

}

\newenvironment{symbolfootnotes}{\def\thefootnote{\fnsymbol{footnote}}}{}

%%%%%%%%%%%%%%%%%%%%%%%%%%%%%%%%%%%%%%%%%%%%%%%%%%%%%%%%%%%%%%%%

\oddsidemargin 0in
\evensidemargin 0in
\marginparwidth 40pt
\marginparsep 10pt
\topmargin 0pt
\headsep 0in
\headheight 0in
\textheight 8.5in
\textwidth 6in
\brokenpenalty=10000

% \handout{number}{date}{title}

\newcommand{\handout}[3]{


\begin{center}
\rule{\textwidth}{.0075in} \\
\rule[3mm]{\textwidth}{.0075in}\\

CMU 17-651\hfill Models of Software Systems\hfill Fall 2018\\[3ex]

{\Large\bf #3}\\[3ex]

Dario A Lencina-Talarico \hfill {\bf Handout #1} \hfill #2

\rule{\textwidth}{.0075in} \\
\rule[3mm]{\textwidth}{.0075in} \\
\end{center}

}

% \homework{number}{date}{title}{due-date}
\newcommand{\homework}[4]{

\begin{center}
\rule{\textwidth}{.0075in} \\
\rule[3mm]{\textwidth}{.0075in}\\

CMU 17-651\hfill Models of Software Systems\hfill Fall 2018\\[3ex]

{\Large\bf #3} \\[3ex]

Dario A Lencina Talarico \hfill  #1  \hfill Due: #2\\

\rule{\textwidth}{.0075in} \\
\rule[3mm]{\textwidth}{.0075in} \\
\end{center}

%\noindent
%{\bf Due date: #4}

}

% \solutionset{number}{date}{title}{due-date}
\newcommand{\solutionset}[4]{

\begin{center}
\rule{\textwidth}{.0075in} \\
\rule[3mm]{\textwidth}{.0075in}\\

CMU 17-651\hfill Models of Software Systems\hfill Fall 2016\\[3ex]

{\Large\bf #3} \\[3ex]

Garlan  \hfill  Solutions for Homework #1  \hfill  #2\\

\rule{\textwidth}{.0075in} \\
\rule[3mm]{\textwidth}{.0075in} \\
\end{center}

%\noindent
%{\bf Due date: #4}

}

% \problem{problem-number}
\newcommand{\problem}[1]{
\vspace{2ex}
\noindent
{\bf Problem #1.}

}

% \solution{solution-number}{points}
\newcommand{\solution}[2]{
\vspace{3ex}
\noindent
{\bf Problem #1}  (#2 points)

}

\newcommand{\cscomment}{
\vspace{1ex}
\noindent Comments: }

% \parts{part-alphabet}{points}
\newcommand{\parts}[2]{
\vspace{2ex}
\noindent
{\bf (#1)}  (#2 points)

}

% \problems{problems-number}{points}
\newcommand{\problems}[2]{
\vspace{3ex}
\noindent
{\bf Problem #1}  (#2 points)

}

\newenvironment{symbolfootnotes}{\def\thefootnote{\fnsymbol{footnote}}}{}

%%%%%%%%%%%%%%%%%%%%%%%%%%%%%%%%%%%%%%%%%%%%%%%%%%%%%%%%%%%%%%%%

\oddsidemargin 0in
\evensidemargin 0in
\marginparwidth 40pt
\marginparsep 10pt
\topmargin 0pt
\headsep 0in
\headheight 0in
\textheight 8.5in
\textwidth 6in
\brokenpenalty=10000

% \handout{number}{date}{title}

\newcommand{\handout}[3]{


\begin{center}
\rule{\textwidth}{.0075in} \\
\rule[3mm]{\textwidth}{.0075in}\\

CMU 17-651\hfill Models of Software Systems\hfill Fall 2018\\[3ex]

{\Large\bf #3}\\[3ex]

Dario A Lencina-Talarico \hfill {\bf Handout #1} \hfill #2

\rule{\textwidth}{.0075in} \\
\rule[3mm]{\textwidth}{.0075in} \\
\end{center}

}

% \homework{number}{date}{title}{due-date}
\newcommand{\homework}[4]{

\begin{center}
\rule{\textwidth}{.0075in} \\
\rule[3mm]{\textwidth}{.0075in}\\

CMU 17-651\hfill Models of Software Systems\hfill Fall 2018\\[3ex]

{\Large\bf #3} \\[3ex]

Dario A Lencina Talarico \hfill  #1  \hfill Due: #2\\

\rule{\textwidth}{.0075in} \\
\rule[3mm]{\textwidth}{.0075in} \\
\end{center}

%\noindent
%{\bf Due date: #4}

}

% \solutionset{number}{date}{title}{due-date}
\newcommand{\solutionset}[4]{

\begin{center}
\rule{\textwidth}{.0075in} \\
\rule[3mm]{\textwidth}{.0075in}\\

CMU 17-651\hfill Models of Software Systems\hfill Fall 2016\\[3ex]

{\Large\bf #3} \\[3ex]

Garlan  \hfill  Solutions for Homework #1  \hfill  #2\\

\rule{\textwidth}{.0075in} \\
\rule[3mm]{\textwidth}{.0075in} \\
\end{center}

%\noindent
%{\bf Due date: #4}

}

% \problem{problem-number}
\newcommand{\problem}[1]{
\vspace{2ex}
\noindent
{\bf Problem #1.}

}

% \solution{solution-number}{points}
\newcommand{\solution}[2]{
\vspace{3ex}
\noindent
{\bf Problem #1}  (#2 points)

}

\newcommand{\cscomment}{
\vspace{1ex}
\noindent Comments: }

% \parts{part-alphabet}{points}
\newcommand{\parts}[2]{
\vspace{2ex}
\noindent
{\bf (#1)}  (#2 points)

}

% \problems{problems-number}{points}
\newcommand{\problems}[2]{
\vspace{3ex}
\noindent
{\bf Problem #1}  (#2 points)

}

\newenvironment{symbolfootnotes}{\def\thefootnote{\fnsymbol{footnote}}}{}

\newtheorem{defn}{Definition}
\newenvironment{spec}{
 \vspace*{8pt}
 \begin{center}
 \begin{minipage}{5in}
 \renewcommand{\baselinestretch}{1}
 }{
 \end{minipage}
 \end{center}
 \vspace*{8pt}
}

\begin{document}

\handout{6}{25 September 1995}{Reasoning About State Machines}
\begin{symbolfootnotes}
\footnotetext[0]{\copyright 1995 Jeannette M. Wing} 
\end{symbolfootnotes}

Given a state machine model of a system, we can do some formal
reasoning about properties of the model.
It's important to remember
that we are proving some property about the {\em model of the system},
not the {\em system} itself.
If the model is ``incorrect'' then we may not be able to prove
anything useful about the system.
Worse, if the model is ``incorrect'' then we may
be able to prove something
that has no correspondence to the real system.  However, we
hope that we've modeled our system properly so that whatever we prove
about our model is true of the system being modeled.

But then, why not just reason about the system itself directly?
One reason is that it's often impossible to reason about the system
itself because it's too large, too complex, or too unwieldy.
Another is that we may be interested in one aspect of the
system
and want to abstract from the irrelevant aspects.
Another is that we may not actually have a real system; our model
could simply be a high-level design of a system we might build
and we want to do some reasoning about our design before
spending the dollars building the real thing.
Another is that it may be impossible to get our hands on the system
(maybe it's proprietary).  Another is that it may be impossible
for us to run the system to check for the property because
of its safety-critical side-effects (like setting off a bomb).
So, in some cases, the best we can do is reason about a model of the
system,
and not the system itself.

In this handout I'll discuss a few kinds of properties we might
want to reason about a state machine model.\footnote{
In this handout, I'm going to be even less precise in my notation
and use of formalism
for the sake of clarity of presentation.}   The most important
of these is {\em invariant properties}, properties that are
true of every reachable state in the system.

\section{Invariants}

An {\em invariant} is a predicate that is true in all states.
In the context of state machines,
we usually care that an invariant is true
in all {\em reachable} states.
The statement of an invariant, $\theta$, in full generality looks like:

\begin{center}
$\forall e: executions \bullet \forall s: S \bullet s ~{\bf in}~ e \Rightarrow \theta(s)$
\end{center}

\noindent where $\theta$ is a predicate over variables in $s$
and {\bf in} is a predicate that says whether a state is in an execution.
Normally the universal quantification over all executions and the
condition that $s$ be in $e$ is omitted (it is understood):

\begin{center}
$\forall s: S \bullet \theta(s)$
\end{center}

\noindent Sometimes we also
omit the universal quantification over $s$ as well because it's also
understood:

\begin{center}
$\theta(s)$
\end{center}

For example, here is an invariant for the Counter example
given in State Machine Variations handout:

\begin{center}
$s(x) \geq 0$
\end{center}

\noindent which says that in all states, $x$'s value is greater than
or equal to 0.  We know this is true because initially $x$'s value
is 0 and because the {\em inc} action always increases $x$'s value
by 1.  Since {\em inc} is Counter's only action, there's no other way
to change $x$.

\subsection{Proving an Invariant}
How do we show that a predicate is an invariant?  There are lots of techniques.
If the state machine is finite, you can do an exhaustive case
analysis and show that
it holds for every state.  This technique is fine if there are a small
number of states or if you have a tool called a {\em model checker}
handy (you will in your Analysis core course next semester).

If your state machine is infinite, you must resort to something else.
I'm going to sketch out three techniques.  They can all also be used
if your state machine is finite.  Technique C is the one most often used
in practice.

\subsubsection*{A. Use induction over states in executions.}

When you have to reason about an infinite domain, the
technique
that should spring to mind is {\em induction}.
Induction is especially appropriate
when there is a natural, often recursive, structure to your domain.
Since what we want to prove is that a property is true of
every state of every execution of the state machine, then I can
induct over the states in the sequence of states of every execution.
Recall that an execution looks like:

\begin{center}

$\langle s_0,  a_1, s_1, a_2,  \ldots,  s_{i-1}, a_i,  s_i, \ldots \rangle $ 

\end{center}

\noindent Then to prove a property, $\theta$, is invariant requires
that for every execution:
\begin{enumerate}
\item Base Case: I show it
holds in the initial state $s_0$, and

\item Inductive Step: I assume it holds in state $s_{i-1}$
and show that it holds in state $s_i$.
\end{enumerate}

\subsubsection*{B. Show that your state space predicate implies the invariant.}

A second technique is that if you're lucky the predicate that you've
used to define your state space is stronger than the invariant you're
trying to prove.  So regardless of whether a state is reachable or
not,
you can prove the invariant holds:

\begin{center}
$P \Rightarrow \theta$
\end{center}

\noindent where $P$ is the predicate describing your set of states.
If it's true of every state, then certainly it's true of every
reachable state.

For example, recall in the Counter example, the predicate $P$ is simply
$ s(x) \geq 0 $.  Hence we can trivially show the invariant property
holds:

\begin{center}
$ s(x) \geq 0 \Rightarrow s(x) \geq 0 $
\end{center}

\subsubsection*{C. Use a proof rule using pre-/post-condition specifications.}

Technique A requires that we reason in terms of first principles---
using structural induction
(over states in executions)---which can sometimes be cumbersome.
And, usually, of course, we are not so lucky as in Technique B.
Technique C is
an alternative inductive proof strategy
that is usually more manageable than Technique A:
we do a case analysis of all
actions, which indirectly proves the same thing as in Technique A.

At the bottom of p. 1
when I argued that the Counter's invariant holds,
we need to show that the invariant holds in each initial state
and then for each action\footnote{Here again, is another good reason
to have only a {\em finite} set of actions.} show that if the
invariant holds in its pre-state, it holds in the post-state.
In general, we have a proof rule that looks like:
\newpage
\begin{center}

$\forall s: I \bullet \theta(s)$\\
\underline{$\forall a: A \bullet \forall s, s': S \bullet (s, a, s') \in \delta \Rightarrow ( (\Phi(s) \wedge \theta(s) \wedge \Psi(s, s')) \Rightarrow \theta(s')) $}\\
$\forall s: S \bullet \theta(s)$

\end{center}

\noindent where $I$ is the set of initial states and $A$ is the set of
actions.
$\Phi$ and $\Psi$ are the pre- and post-conditions of $a$, respectively.
Or, said in English:

\begin{enumerate}
\item Show that $\theta$ is true for each initial state.

\item For each action,
\begin{itemize}

\item assume
\begin{itemize}
\item the pre-condition $\Phi$ holds in the pre-state,
\item the invariant $\theta$ holds in the pre-state, and
\item the post-condition $\Psi$ holds in the pre- and post-states, then
\end{itemize}

\item show
\begin{itemize}
\item $\theta$ holds in the post-state.
\end{itemize}

\end{itemize}

\item Conclude that $\theta$ is an invariant.
\end{enumerate}

The two main proof steps are sometimes called
(1) {\em establishing the invariant} (true in initial states)
and (2) {\em preserving the invariant} (assuming it's true in
a pre-state and showing that each action's the post-state preserves it).

What's the rationale for this proof rule?  First, notice I care only
about reachable states (that's why the $\delta$ appears above).
Second, consider any execution of my state machine:

\begin{center}

$\langle s_0,  a_1, s_1, a_2,  \ldots,  s_{i-1}, a_i,  s_i, \dots \rangle $ 

\end{center}

\noindent As in Technique A I need to make sure that $\theta$ holds in $s_0$
(establishing the invariant).  Also,
for any pair of successive states, $(s_{i-1}, s_i)$, in
the
execution, if I assume $\theta$ holds in $s_{i-1}$ then I need to show
it holds in $s_i$.  Since the only way I can get from any $s_{i-1}$ to
the next state $s_i$ is by one of the actions $a$ in $A$, then if I
show the invariant if preserved for each $a$, I've shown for each
reachable state
the
invariant holds.

\subsection{OddCounter}

Let OddCounter be the state machine:

\centerline{\epsfbox{pictures/oddcounter.ps}}

\noindent More precisely,
\begin{verse}
OddCounter = $($ \\
$\{ s: \{x\} \rightarrow int \},$ \\
$\{ s: \{x\} \rightarrow int | s(x) = 1 \},$ \\
$\{inc(i: int)\},$ \\
$\delta$ = \\

\begin{spec}
\begin{tabbing}
\indent {\em inc}\={\em(i: int)}\\
             \>{\bf pre} $i$ is even\\
             \>{\bf post} $x' = x + i$
\end{tabbing}
\end{spec}

$)$.
\end{verse}

The invariant I want to prove is that OddCounter's
state always holds an odd integer:
\begin{center}
$\theta \equiv$ $x$ is odd.
\end{center}

Intuitively we see this
is true because
\begin{enumerate}
\item It's true of the initial state (1 is an odd integer).
\item  For the action {\em inc}:
\begin{itemize}
\item I assume:
\begin{itemize}
\item $i$ is even (i.e., the pre-condition holds in the pre-state),
\item $x$ is odd (i.e., the invariant holds in the pre-state), and
\item $x' = x + i$ (i.e., the post-condition holds in the pre- and post-states)
\end{itemize}
\item I need to show that $x'$ is odd.
\begin{itemize}
\item This is true since
from facts about numbers I know an odd number ($x$) plus an even number ($i$) is always odd.
\end{itemize}
\end{itemize}
\end{enumerate}


\subsection{Fat Sets}
This example has two purposes.
One is
to give another example of an invariant.  The other is
to hint at how state machines are appropriate models of abstract
data types, as you might implement in your favorite object-oriented language.

Here is a description of a FatSet abstract data
type modeled as a state machine:
\begin{itemize}
\item$S = \{ s: \{t\} \rightarrow set[int] \}$.
The state variable, $t$, maps to a set of integers.

\item $I = \{ s: \{t\} \rightarrow set[int] | \#s(t) = 1 \}$.
The set of initial states is the set of states that
map $t$ to a singleton set.  Notice that there are an infinite number
of initial states.

\item {\em A = $\{$ union(u: set[int])/ok(), card()/ok(int)} $\}$

\item $\delta$ =

\begin{spec}
\begin{tabbing}
\indent {\em union}\={\em(u: set[int])/ok()}\\
             \>{\bf pre} $u \neq \emptyset$\\
             \>{\bf post} $t' = t \cup u$\\
 \\
\indent {\em card}{\em()/ok(int)}\\
             \>{\bf pre} $true$\\
             \>{\bf post} $result = \#t \wedge t' = t $
\end{tabbing}
\end{spec}

\end{itemize}

Suppose I want to show the property that the size of the
FatSet $t$ is always greater than or equal to 1:

\begin{center}
$\theta \equiv \#s(t) \geq 1$
\end{center}

Here's an informal proof:

\begin{enumerate}
\item It's true of all initial states since the size of all singleton sets
is 1.
\item  We need to show the invariant is preserved for each of {\em
union}
and {\em card}.
\begin{enumerate}
\item ({\em union}). Assume
\begin{itemize}
\item $u$ is nonempty (i.e., the pre-condition holds in the pre-state),
\item $t$'s size is $\geq 1$  (i.e., the invariant holds in the pre-state), and
\item $t' = t \cup u$ (i.e., the post-condition holds in the pre- and post-states)
\end{itemize}
Then I need to show that the size of $t'$ is $\geq 1$.  This is true since
taking the union of two non-empty sets ($t$ and $u$) is a non-empty
set,
whose size is $\geq 1$.

\item ({\em card}). Assume
\begin{itemize}
\item $t$'s size is $\geq 1$ (i.e., the invariant holds in the pre-state), and
\item $result = \#t \wedge t' = t$ (i.e., the post-condition holds in the pre- and post-states)
\end{itemize}
Then I need to show that the size of $t'$ is $\geq 1$.  This is true since
$t' = t$ and $t$'s size is $\geq 1$.
\end{enumerate}
\end{enumerate}

\noindent Notice how awful it would be if I had to write out these proof steps
in gory detail!

\vspace{.15in}
Two points about this example:
\begin{itemize}
\item You should notice that
the only interesting part of the proof above was for {\em union},
the only action that changes the value of any state variable.
An action that changes the state of some state variable is called
a {\em mutator}.
I didn't really have anything to prove for {\em card} because
the action does not change the state of any state variable.  I'll call
such an action a {\em non-mutator}.
In practice, we need to show only for mutators that the invariant
is preserved because non-mutators by definition
cannot change state.  So if the invariant holds before the non-mutator
is called (pre-state); then it holds afterwards (post-state).

\item Second I deliberately chose a value space for $t$ that
includes ``unreachable values.''  In particular, $t$ can never
be the empty set because it starts out non-empty and always remains
non-empty.  But, remember, I need to show the invariant
holds of only reachable states.
\end{itemize}

Now, suppose I added a {\em delete} action to my FatSet example.
Let {\em delete} have the following behavior:

\begin{spec}
\begin{tabbing}
\indent {\em delete}\={\em (i:int)/ok()}\\
            \>{\bf pre} $t \neq \emptyset$\\
            \>{\bf post} $t' = t\setminus\{i\}$
\end{tabbing}
\end{spec}

\noindent Then my invariant would no longer hold because
if {\em delete} were called in any state where $t = \{i\}$
(where $i$ is the argument to {\em delete})
then the size of $t'$ would be 0.

\subsection{Diverging Counter}
This example has two purposes.
One is
to give another example of an
invariant.  The other is
to give an example of a state machine with more
than one state variable.  The invariant property captures an invariant
relationship that we want to maintain between these two state variables.

Here's a two-integer counter with state variables, $x$ and $y$, whose
values get further and further apart as you {\em poke} the machine.

\centerline{\epsfbox{pictures/divergingcounter.ps}}


Here's a description of the DivergingCounter:

\begin{verse}
DivergingCounter = $($ \\
$\{ s: \{x, y\} \rightarrow int \},$ \\
$\{ s: \{x, y\} \rightarrow int | s(x) = -s(y) \},$ \\
$\{poke(i: int)\},$ \\
$\delta$ = \\

\begin{spec}
\begin{tabbing}
\indent {\em poke}\={\em (i: int)}\\
           \> {\bf pre} $i > 0$\\
           \> {\bf post} $x' = x + i \wedge y' = y - i$
\end{tabbing}
\end{spec}

$)$.
\end{verse}

\noindent Notice that all the states drawn above are legitimate initial states.
So is the state where $x$ and $y$ are both initialized to 0.

The invariant maintained by DivergingCounter is:

\begin{center}
$\theta \equiv s(x) + s(y) = 0$
\end{center}

\noindent Can you prove it?

\subsection{Metacomment}

In invariants it's awkward writing $s(x)$ and $s(y)$ all
the time since $s$ is universally quantified.  So more typically
you'll see invariants expressed directly in terms of the state
variables.
For the DivergingCounter, you'd more likely see its invariant in this
more readable format:

\begin{center}
$\theta \equiv x + y = 0$
\end{center}

\noindent We use this syntactic sugar
in the pre- and post-conditions already.
\section{Constraints}

An invariant is supposed to be true of every {\em state}
of every execution of a state machine.  You might ask is there
a corresponding notion for {\em state transitions}?  The answer is
``yes,''
though because it is not as
commonly discussed in the literature, there is no common term
for such a property.  Some people might simply call it another
kind of invariant, one over state transitions rather than states.
To avoid confusion with state invariants,
I'm going to call it a {\em constraint}.  
This word is not standard; also,
others may use ``constraint'' to mean something different.

Consider any execution of a state machine:

\begin{center}

$\langle s_0, a_1, s_1, a_2, \dots, s_{i-1}, a_i, s_i, a_{i+1}, s_{i+1}, \dots, s_{j-1}, a_j, s_j, \ldots \rangle$ 

\end{center}

A {\em constraint} is a predicate that is true in all pairs of states,
$s_i$ and $s_j$, in every execution, where $s_j$
follows $s_i$, but need not immediately follow it ($s_j$ does not have
to be $s_{i+1}$).

The statement of a constraint, $\chi$, in full generality looks like:

\begin{center}
$\forall e: executions \bullet \forall s_i, s_j: S \bullet (s_i ~{\bf in}~ e \wedge s_j ~{\bf in}~ e \wedge i < j) \Rightarrow \chi(s_i, s_j)$
\end{center}

\noindent As for statements of invariants, we omit (because it's
understood) the
universal
quantification over executions and states; the condition about $s_i$
and $s_j$ both being states in the execution; and the condition that $s_i$
precedes $s_j$ in the execution.

A constraint that holds for the Counter example
is that its value always strictly increases:

\begin{center}
$s'(x) > s(x)$
\end{center}

Does this constraint (appropriately restated) hold for SimpleCounter?
BigCounter (careful!)? FatCounter?  RandomCounter?

\subsection{Proving Constraints}
If the constraint we are interested in proving is a ``transitive'' property
(if it holds for $s_i$ and $s_{i+1}$ and for $s_{i+1}$ and $s_{i+2}$
then it holds for $s_i$ and $s_{i+2}$) then
we can use the following proof rule to show the constraint holds
of any execution of the state machine:

\begin{center}
\underline{$\forall a: A \bullet \forall s, s': S \bullet (s, a, s') \in \delta
\Rightarrow ( \Phi(s) \wedge \Psi(s,s') \Rightarrow \chi(s, s') )$}\\
$\chi(s_i, s_j)$
\end{center}

\noindent where $A$ is the set of actions and $s_i$ and $s_j$ are quantified and qualified as
described
above.  Or, in English:

\begin{enumerate}
\item For each action $a \in A$,
\begin{itemize}
\item assume
\begin{itemize}
\item $\Phi$ holds in the pre-state,
\item $\Psi$ holds in the pre/post-states, and
\end{itemize}

\item
show
\begin{itemize}
\item $\chi$ holds in the pre/post-states.
\end{itemize}
\end{itemize}

\item Conclude that $\chi$ holds of all pairs of states in all executions.
\end{enumerate}

What's the rationale for this proof rule?  First, again I care only
about reachable states.
Second, consider any execution of my state machine:

\begin{center}

$\langle s_0, a_1, s_1, a_2, \dots, s_{i-1}, a_i, s_i, a_{i+1}, s_{i+1}, \dots, s_{j-1}, a_j, s_j, \ldots \rangle$ 

\end{center}

\noindent
If I can show that $\chi$ holds over any pair of successive states,
$(s_i, s_{i+1})$,
i.e., every single state transition, then surely it holds over any
pair of states, ($s_i, s_j$), where $i < j$.
To show that it holds in any pair of successive states, I need
only consider every possible action, which is the only way I can
get from $s_i$ to $s_{i+1}$.  For each action, I need to make
sure that the conjunction of its pre- and post-condition predicates
imply the constraint.

\subsection{Fat Sets Again}

Here is a constraint for my FatSet example:

\begin{center}
$\chi \equiv \forall x: int \bullet x \in s_i(t) \Rightarrow x \in s_j(t)$
\end{center}

\noindent where $s_i$ and $s_j$ are implicitly defined and qualified
as usual.  This says that once an integer gets added to my set $t$ it never
disappears.  We know this constraint holds because there is no way
to delete elements from the set.

Notice that saying that the cardinality of $t$ always
strictly increases:

\begin{center}
WRONG! $\chi \equiv \#s_i(t) > \#s_j(t)$
\end{center}

\noindent is not a constraint for FatSet.
It does not hold since taking the union of two sets
may not necessarily increase the size of either.

\subsection{MaxCounter}

Constraints are useful for stating succinctly when things
do not change in value.
Consider the following MaxCounter machine whose state variable $x$
can never exceed the value of the other state variable {\em max}.
{\em Max} is initialized to 15 and
its value never changes.

\begin{verse}
MaxCounter = $($ \\
$\{ s: \{x, max\} \rightarrow int  \},$ \\
$\{ s: \{x, max\} \rightarrow int | s(x) = 0 \wedge s(max) = 15 \},$ \\
$\{ inc(i: int) \},$ \\
$\delta$ = \\

\begin{spec}
\begin{tabbing}
\indent {\em inc}\={\em (i: int)}\\
           \> {\bf pre} $ x + i \leq max$\\
           \> {\bf post} $x' = x + i \wedge max' = max$
\end{tabbing}
\end{spec}

$)$.
\end{verse}

It's trivial to show the following constraint:

\begin{center}
$\chi \equiv s_i(max) = s_j(max)$
\end{center}

This kind of example may look simplistic but it generalizes
to any system where you want to ensure that some state
variable never changes.  When we have a huge state
space (as is more typical of software systems
than in my zillions of simple counter examples),
very often we are careful to state how some state
variable changes but forget to say what state
variables do not change.  Constraints are a nice way
to describe those properties.

\section{Other Properties of State Machines}

The kinds of properties you have seen so far are sometimes called {\em
safety} properties, properties that say that ``nothing bad
happens.''  Another class of properties that people often discuss are
called {\em liveness} properties, properties that say that
``something good eventually happens.''  For a sequential system,
computing the correct answer is an example of a safety property and
termination of a program is an example of a liveness property (this
program ``eventually'' terminates).  For a concurrent system, deadlock
freedom and mutual exclusion are examples of safety properties; for a
distributed system, the property that a message sent is eventually
received is an example of a liveness property.

For concurrent and distributed systems, there are many interesting
liveness properties so we will defer discussing them until later
in the course.

\end{document}











