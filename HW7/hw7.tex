\documentclass{article}
\usepackage{zed-csp}
\usepackage{amsfonts}
\input{handout}
\newenvironment{spec}{
 \vspace*{8pt}
 \begin{center}
 \begin{minipage}{5in}
 \renewcommand{\baselinestretch}{1}
 }{
 \end{minipage}
 \end{center}
 \vspace*{8pt}
}
\begin{document}

\homework{}{15 October 2018}{Homework \#7: Invariants and Introduction to Z}{}

\noindent \textbf{Part 1: Invariants}\\

\noindent Consider the Diverging Counter example of Chapter 10 of GWC09. Prove that $x+y=0$ is an invariant of the $DivergingCounter$ state machine.\\
\\
DivergingCounter = $($ \\
$[ x, y : \mathbb{Z} ],$ \\
$\{ s: [x, y : \mathbb{Z}] | s(x) = -s(y) \},$ \\
$\{poke(i: \mathbb{Z})\},$ \\
$\delta$ == \\

\begin{spec}
\begin{tabbing}
\indent {\em poke}\={\em (i: $\mathbb{Z}$)}\\
           \> {\bf pre} $i > 0$\\
           \> {\bf post} $x' = x + i \wedge y' = y - i$
\end{tabbing}
\end{spec}

$)$.
\\
1. Base case: show that $\theta$ holds in the initial state. \\
Here there's only one initial state: \\
$[x = 0;y = 0 ]$ \\ 
Proof: \\
\begin{tabular}{l ll lll}
     x + y = 0 \\ &    &  \\
     = &   & [initial state] \\
     0 + 0 = 0 & & \\
     = &   & [aritmetic] \\
     0 = 0  \\
\end{tabular} \\
2. Induction step on inc: \\
\\
Show: $\theta (s), pre(s), post(s, s') \vdash \theta(s')$ \\
That is, from $x' = x + i \wedge y' = y - i$ \\
\\
$ \theta (s) == x + y = 0 $ \\
prove that $ x' + y' = 0 $ \\
Proof: \\
\begin{tabular}{l ll lll}
     y' \\ &    &  \\
     = &   & [poscondition] \\
     y - i & & \\
     = &   & [introduction hypothesis y + x = 0, or y = -x] \\
     -x - i \\
     = & & [factorizing] \\ 
     -(x + i) \\
     = & & [replacing definition of x'] \\
     -x' \\
     \\
    y' = -x' is equivalent to y' + x' = 0 \\
\end{tabular} \\
\noindent (\textsc{Note:} In your proof use style C (in Section 10.1.1) of reasoning about invariants and a similar degree of formalism as in the lecture on this topic.)

\vspace{12pt}

\noindent \textbf{Part 2: Z} \\

\noindent \textsc{Note}: For this part of the assignment you must format your
answers using \LaTeX and typecheck the answers using {\em fuzz},  Z-EVES, or the Community Z tools.  \\

\noindent Write a Z specification of the following system. Your specification should include sufficient explanatory
prose to make it easily understandable. (The prose is important---answers with little
or no prose will receive a low grade.)\\[2ex]
A teacher wants to keep a register of students in the class, and to record which students have completed their homework. \\[2ex]
Let the given set $Student$ represent the set of all students who might ever be enrolled in a class:
\begin{zed}
[Student]
\end{zed}
Specify each of the following:
\begin{enumerate}
\item The state space for a register.

\textsc{Hint}: use two sets of students:
\begin{schema}{Register}
enrolled: \power Student\\
completed: \power Student
\where
completed \subseteq enrolled \\
\end{schema}
\item An operation to enroll a new student.
\item The initial state(s)  for your state space.
\item An operation to record that a student (already enrolled in class) has completed the homework.
\item An operation to inquire whether a student (who must be enrolled) has completed the homework (the answer is to be either `Yes' or `No').
\item A robust version of the system. (Be sure to use the schema calculus, as illustrated by the class lecture and the paper by Spivey on Z.) \\  
\end{enumerate}

\end{document}
