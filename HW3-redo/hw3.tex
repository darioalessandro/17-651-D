\documentclass{article}
\usepackage{verbatim,indentfirst,amsmath,amsthm,amssymb,latexsym,enumerate,zed-csp}
\usepackage[usenames,dvipsnames]{xcolor}

\newcommand{\revise}[1] {\textcolor{red} {#1}} 				% Text to revise

\input{handout}
\begin{document}

\homework{}{17 September 2018}{Homework \#3:
Proof}

\noindent {\bf INSTRUCTIONS}: before completing your homework,  make sure you have considered the following:
\begin{itemize}
\item You are allowed to use only primitive inference rules.
\item Remember to include vertical lines to represent the scope any assumptions.
\item Remember that all assumptions in your proofs must be discharged.
\item Double-check that your line references are correct when applying inference rules.
\end{itemize}

\bigskip

\begin{enumerate}[\bf I.]

\item \textbf{Proof: Propositional Logic} \\[6pt]
In this section {\em use only the primitive inference rules} of
propositional calculus.
\begin{enumerate}[1.] \setcounter{enumii}{0}
\item Provide derivations for each of the following, using natural
deduction:
\begin{enumerate}[a.]
 %\item $(p \land q)\land r\vdash p \land (q \land r)$ \\
 \item $p \land q,p\implies s,q\implies t\vdash s\land t$ \\
   \begin{tabular}{l ll lll}
     1. & $p \land q$ & premise \\
     2. & $p \implies s$ & premise \\
     3. & $q \implies t$ & premise \\
     4. & $p$            & $\land$-elim, 1 \\
     5. & $s$            & modus ponens, 2,4 \\
     6. & $q$            & $\land$-elim, 1 \\
     7. & $t$            & modus ponens, 3,6 \\
     8. & $s\land t$     & $\land$-intro, 5,7 \\
   \end{tabular} \\
 \item $q \implies \neg p,p\land q\vdash r$ \\
   \begin{tabular}{l ll lll llll}
     1. & $q \implies \neg p$ & premise \\
     2. & $p \land q$         & premise \\
     3. & $q$                 & $\land -elim, 2$ \\
     4. & $\neg p$            & modus ponens, 1 \\  
     5. & $p$                 & $\land -elim, 2$ \\
     6.	& $\neg r$            & assumption  & l\\ 
     7. & $\neg p$            & Copy from 4 & l\\
     8. & $p$                 & Copy from 5 & l\\
     4. & $r$                 & $\neg -elim$, 6-8 \\
   \end{tabular} \\
 \item $p \land q \vdash p \implies q$ \\
   \begin{tabular}{l ll lll llll}
     1. & $p \land q$ & assuption & l\\
     2. & $p$         & $\land-elim$,1 & l\\
     3. & $q$         & $\land-elim$,1 & l\\
     4. & $p \implies q$ & $\implies-intro$,1-3,2,3 \\
   \end{tabular} \\
 \item $\neg\neg q \vdash q \lor r$ \\
   \begin{tabular}{l ll lll}
    1. & $\neg\neg q$ & assuption & l\\
    2. & $\neg q$ & $\neg-elim, 1$ & l \\
    3. & $q$ & $\neg-elim, 2$ & l\\
    4. & $q \lor r$ & $\lor-intro, 1-3$\\
   \end{tabular} \\
 \item $p \implies (q \land r), \neg p \implies r, p \lor \neg p \vdash r$ \\
   \begin{tabular}{l ll lll llll }
    1. & $\neg p \implies r$       & premise \\
    2. & $p \lor \neg p$           & assumption       &   & l  \\
    3. & $p$                       & $\lor$-elim, 2   &   & l  \\
    4. & $p \implies (q \land r) $ & assumption       & l & l  \\
    5. & $(q \land r)$             & modus ponens,4,3 & l & l  \\
    6. & $r$                       & $\land-elim$, 2-5 \\
   \end{tabular} \\
\end{enumerate}
\end{enumerate}

\vspace{12pt}

\item \textbf{Proof: Predicate Logic}\\[6pt]
In this section {\em use only the primitive inference rules} of predicate
calculus.
\begin{enumerate}[1.] \setcounter{enumii}{1}
\item Show using natural deduction:
\begin{enumerate}[a.]
\item $\forall x:T\bullet P(x)\land Q(x)\dashv
  \vdash (\forall x:T\bullet P(x))\land (\forall y:T\bullet Q(y))$ \\
  \\
  $\forall x:T\bullet P(x)\land Q(x)
  \vdash (\forall x:T\bullet P(x))\land (\forall y:T\bullet Q(y))$ \\
  
  \begin{tabular}{l ll lll llll}
    1. $\forall x:T\bullet P(x)\land Q(x)$                        & assumption               &   &   & l \\
    2. $x \in T$                                                  & assumption               &   & l & l \\
    3. $P(x)\land Q(x)$                                           & $\forall-elim$, 1        &   & l & l \\
    4. $P(x)$                                                     & $\land-elim$, 3          &   & l & l \\
    5. $Q(x)$                                                     & $\land-elum$, 3          &   & l & l \\ 
    6. $y \in T$                                                  & assumption               & l & l & l \\
    7. $Q(y)$                                                     & transforming x to y, 5,6 & l & l & l \\ 
    8. $\forall y:T\bullet Q(y)$                                  & $\forall-intro$ 6,7      &   & l & l \\
    9. $\forall x:T\bullet P(x)$                                  & $\forall-intro$ 2,4      &   & l & l \\
    10. $\forall x:T\bullet P(x))\land (\forall y:T\bullet Q(y))$ & $\land-intro$ 8,9        &   &   &   \\
  \end{tabular} \\
  \\
  $(\forall x:T\bullet P(x))\land (\forall y:T\bullet Q(y))
  \vdash \forall x:T\bullet P(x)\land Q(x)$ \\
  \\
  \begin{tabular}{l ll lll llll}
    1. $(\forall x:T\bullet P(x))\land (\forall y:T\bullet Q(y))$ & assumption              &   &   & l \\
    2. $(\forall x:T\bullet P(x)$                                 &                         &   &   & l \\
    3. $(\forall y:T\bullet Q(y))$                                &                         &   &   & l \\
    4. $a \in T$                                                  & assumption              &   & l & l \\
    5. $b \in T$                                                  & assumption              & l & l & l \\
    6. $P(a)$                                                     & $\forall-elim,2,4$      & l & l & l \\
    7. $Q(b)$                                                     & $\forall-elim,3,5$      & l & l & l \\
    8. $P(a)\land Q(b)$                                           & $\land-intro,6,7$       & l & l & l \\
    9. $\forall x:T\bullet P(x)\land Q(x)$                        & $\forall-intro, 4-8$    &   &   &   \\
   \end{tabular} \\
\item $\exists x:T\bullet P(x) \lor Q(x) \dashv \vdash
  (\exists x:T \bullet P(x)) \lor (\exists x:T \bullet Q(x))$ \\
  \\
  $\exists x:T\bullet P(x) \lor Q(x) \vdash
  (\exists x:T \bullet P(x)) \lor (\exists x:T \bullet Q(x))$ \\
  \begin{tabular}{l ll lll llll}
    1. $\exists x:T\bullet P(x) \lor Q(x)$                           & assumption           &   & l \\
    2. $a \in T \land P(a) \lor Q(a)$                                & assumption           & l & l \\
    3. $P(a) \lor Q(a)$                                              & $\land-elim,2$       & l & l \\
    4. $P(a)$                                                        & $\lor-elim, 3$       & l & l \\
    5. $Q(a)$                                                        & $\lor-elim, 3$       & l & l \\
    6. $a \in T$                                                     & $\land-elim,2$       & l & l \\
    6. $\exists x:T\bullet P(x)$                                     & $\exists-intro,6,4$  & l & l \\
    7. $\exists x:T\bullet Q(x)$                                     & $\exists-intro,6,5$  & l & l \\
    8. $(\exists x:T \bullet P(x)) \lor (\exists x:T \bullet Q(x))$  & $\lor-intro, 6,7$    &   &   \\
  \end{tabular} \\    
  
  $(\exists x:T \bullet P(x)) \lor (\exists x:T \bullet Q(x))
    \vdash \exists x:T\bullet P(x) \lor Q(x)$ \\
  \begin{tabular}{l ll lll llll}
    1. $(\exists x:T \bullet P(x)) \lor (\exists x:T \bullet Q(x))$  & assumption           &   &   & l \\
    2. $\exists x:T \bullet P(x)$                                    & $\lor-elim, 1$       &   &   & l \\
    3. $\exists x:T \bullet Q(x)$                                    & $\lor-elim, 1$       &   &   & l \\
    4. $a \in T \land P(a)$                                          & assumption           &   & l & l \\
    5. $P(a)$                                                        & $\land-elim$ 4       &   & l & l \\
    6. $b \in T \land Q(b)$                                          & assumption           & l & l & l \\
    7. $Q(b)$                                                        & $\land-elim$ 6       & l & l & l \\    
    8. $b \in T$                                                     & $\land-elim$ 6       & l & l & l \\
    9. $P(a) \lor Q(b)$                                              & $\lor-intro, 2-4$    & l & l & l \\
    10. $\exists x:T\bullet P(x) \lor Q(x)$                          & $\exists-intro,8,9$  &   & l & l \\
    11. $\exists x:T\bullet P(x) \lor Q(x)$                          & $\exists-elim, 2,4,10$  &   &   &   \\
    12. $\exists x:T\bullet P(x) \lor Q(x)$                          & $\exists-elim, 3,6-10$  &   &   &   \\
    
   \end{tabular} \\
\end{enumerate}
(\textsc{Note}: $p \dashv\vdash q$ is a shorthand for ``$p \vdash q$
and $q \vdash p$." That is, for $p \dashv \vdash q$ you need to show
two separate derivations: one for $p \vdash q$ and another for $q
\vdash p$.)
\end{enumerate}

\end{enumerate}
\end{document}
