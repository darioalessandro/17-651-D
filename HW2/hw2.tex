\documentclass{article}
\usepackage{verbatim,indentfirst,amsmath,amsthm,amssymb,latexsym,enumerate,zed-csp}
\usepackage[usenames,dvipsnames]{xcolor}

\newcommand{\revise}[1] {\textcolor{red} {#1}} 				% Text to revise

\input{handout}
\begin{document}

    \homework{}{10 September 2018}{Homework \#2: Logic}{}

    \begin{enumerate}[\bf I.]
        \item \textbf{Propositional Logic}

        \begin{enumerate}[1.]
            \item Construct a truth table for each of the following. Include in your tables intermediate expressions needed to construct the final truth tables column.

            \begin{enumerate}[a.]
                \item $p \land (p \lor q)$
                \begin{center}
                    \begin{tabular}{ |c|c|c|c| }
                        \hline
                        p & q & $(p \lor q)$ & $p \land (p \lor q)$ \\
                        false & false & false & false \\
                        true & false & true & true \\
                        false & true & true & false \\
                        true & true & true & true \\
                        \hline
                    \end{tabular}
                \end{center}

                \item $\neg p \land (p \lor (q\implies p))$
                \begin{center}
                    \begin{tabular}{ |c|c|c|c|c| }
                        \hline
                        p & q & $(q\implies p)$ & $(p \lor (q\implies p))$ & $\neg p \land (p \lor (q\implies p))$ \\
                        false & false & true & true & true \\
                        true & false & true & true & false \\
                        false & true & false & false & false \\
                        true & true & true & true & false \\
                        \hline
                    \end{tabular}
                \end{center}
                \item $(p \implies q)\implies(\neg p \lor q)$
                \begin{center}
                    \begin{tabular}{ |c|c|c|c|c| }
                        \hline
                        p & q & $(p \implies q)$ & $(\neg p \lor q)$ & $(p \implies q)\implies(\neg p \lor q)$ \\
                        false & false & true & true & true \\
                        true & false & false & false & true \\
                        false & true & true & true & true \\
                        true & true & true & true & true \\
                        \hline
                    \end{tabular}
                \end{center}
            \end{enumerate}
            \item Which of the above sentences are:
            \begin{enumerate}
                \item valid?: c \\
                Because c is true for all values of p and q.
                \item satisfiable?: a, b, c \\
                Because are true under at least one interpretation of atomic propositions.
                \item contingent?: a, b \\
                Because they are neither tautologies nor contradictions.
                \item inconsistent?: none \\
                Because we have no sentences that are false for all propositions of q and p.
            \end{enumerate}
            Briefly explain why.
            \item Demonstrate using truth tables that the following sentences have the same meaning. Include intermediate expressions, as above.

            %(5 points)
            \begin{itemize}
                \item $p \implies q$
                \item $\neg (p \land \neg q)$
            \end{itemize}
        \end{enumerate}

        \begin{center}
            \begin{tabular}{ |c|c|c|c|c| }
                \hline
                col1 & col2 & col3 & col4 & col5 \\
                p & q & $(p \land \neg q)$ & $\neg (p \land \neg q)$ & $p \implies q$ \\
                false & false & false & true & true \\
                true & false & true & false & false \\
                false & true & false & true & true \\
                true & true & false & true & true \\
                \hline
            \end{tabular}
        \end{center}
        The truth table for col4 and col5 are equal, hence, we demonstrated that $\neg (p \land \neg q)$ has the same meaning than $p \implies q$ for all values of $p$ and $q$.
        \item \textbf{Predicate Logic}
        %(19 points)
        \begin{enumerate}[1.]\setcounter{enumii}{3}
        \item Which occurrences of the variables $x$ and $y$ are free and
        which are bound in each of the following? Briefly explain why.\\
        %(2 points each)
        \textsc{Note}: Recall that a variable may be both
        bound and free in the same sentence. In such cases, explain where in the sentence the variable is bound, and where it is free.
        \begin{enumerate}[a.]
            \item $(\exists y:N\bullet y>2)\wedge(\forall x:N\bullet x+1>x)$ \\
            Both y and x are bounded, y by $\exists y:N$ and x by $\forall x:N$
            \item $x=2\ast y$ \\
            x and y are free because there are no quantifiers.
            \item $(\exists y:N\bullet y>2)\wedge(\forall x:N\bullet x>y)$ \\
            y is bounded in  $(\exists y:N\bullet y>2)$ but is not in $(\forall x:N\bullet x>y)$. \\
            x is bounded where it appears.
            \item $\forall x:N\bullet((\exists y:N\bullet y>x)\wedge x=2\ast y)$ \\
            x is bounded by the initial $\forall x:N$ for the whole expression, but y is free in $x=2\ast y$.
        \end{enumerate}
        \item Translate the following sentences into predicate logic (with equality), using the translation key provided.

        \textsc{Note}: You may only use the standard universal and
        existential quantifiers ($\forall$ and $\exists$). Do \emph{not} use the unique existential quantifier ($\exists !$).
        \begin{center}
            \begin{tabular}{rl}
                $E$: & the set of elephants\\
                $A$: & the set of animals\\
                $G(x)$: & $x$ is green\\
                $E(x)$: & $x$ is an elephant\\
                $N(x,y)$: & the name of $x$ is $y$
            \end{tabular}
        \end{center}
        \begin{enumerate}[a.]
            \item Some elephants are green. \\ %(2 points)
            $\exists x:E\bullet G(x)$
            \item All elephants are green. \\%(2 points)
            $\forall x:E\bullet G(x)$
            \item If an animal is green, it is an elephant. \\%(2 points)
            $\forall x:A\bullet G(x) \implies E(x) $
            \item No green animal is an elephant. \\%(2 points)
            $\forall x:A\bullet G(x) \implies \neg E(x) $
            \item There is exactly one green elephant. \\%(2 points)
            $\exists x:E\bullet G(x) \land \forall y:E\bullet G(y) \implies x = y $
            \item There is \emph{exactly one} green elephant, and his name is James. \\%(3 points)
            $\exists x:E\bullet G(x) \land \forall y:E\bullet G(y) \implies x = y \land N(x,James) $
            \label{james}
        \end{enumerate}

        %%%%% Important note: In years when proof techniques are included with the above problems remove the next problem and cover completely in class and recitation.

        \item Translate the following sentences into predicate logic (with equality), using the translation key provided.

        \textsc{Note}: You may only use the standard universal and existential quantifiers ($\forall$ and $\exists$). Do \emph{not} use the unique existential quantifier ($\exists !$).

        \begin{center}
            \begin{tabular}{rl}
                $S$:            & the set of students\\
                $T$:            & the set of topics, which has $logic$ and $models$ as elements\\
                $MSE(s)$:       & $s$ is an MSE student\\
                $Likes(s,t)$:   & student $s$ likes topic $t$\\
            \end{tabular}
        \end{center}

        \begin{enumerate}[a.]
            \item Some MSE students like logic. \\
            $\exists s:S\bullet MSE(s) \land Likes(s,logic)$
            \item MSE students like logic. \\
            $\forall s:S\bullet MSE(s) \implies Likes(s,logic)$
            \item MSE students like logic, and only logic. \\
            $\forall s:S\bullet (MSE(s) \implies (Likes(s,logic))) \land (\forall t:T \bullet (t \neq logic \implies \neg Likes(s,t)))$
            \item No MSE student likes logic. \\
            $\forall s:S\bullet MSE(s) \implies \neg Likes(s,logic)$
            \item If an MSE student likes logic then he/she likes Models. \\
            $\exists s:S\bullet (MSE(s) \land Likes(s,logic)) \implies Likes(s,Models)$
            \item Exactly one MSE student likes Models. \\
            $\exists s:S\bullet MSE(x) \land \forall y:S\bullet (MSE(y) \land Likes(y,Models))  \implies s = y $
        \end{enumerate}


        \item In this class we will be creating various models of an infusion pump. An infusion pump is a
        device used in hospitals to feed fluids intravenously to patients through one of several ``infusion
        lines." Each line is a physical tube connected to a patient.

        Consider the following excerpt from a description of a typical pump provided to us by the Food and
        Drug Administration:

        \begin{enumerate}[A.]
            \item An infusion line may become pinched causing the flow to be blocked.  This will be recognized
            by the pump as an occlusion and will cause the pump to alarm.

            \begin{enumerate}[i.]
                \item The mitigation is to straighten the line and re-start the pump.
                \item Caregiver may silence the alarm during the procedure.
            \end{enumerate}

            \item The infusion line may become plugged.  The pump will recognize an occlusion and alarm.

            \begin{enumerate}[i.]
                \item The mitigation is to clear the line and re-start the pump.
                \item Caregiver may silence the alarm during the procedure.
            \end{enumerate}

            \item Electrical failure may occur causing the pump to switch to battery operation.

            \begin{enumerate}[i.]
                \item Pump will switch over to battery power and notify the caregiver visually.
                \item Switch may not occur if the battery is not properly charged.
            \end{enumerate}

        \end{enumerate}

        Questions:

        \begin{enumerate}[a.]
            \item Define some sets and predicates appropriate to this domain
            (similar to the elephant problem above). \\
            $P$:  the set of pumps .\\
            $Pinched(x)$: & $x$ is pinched. \\
            $Plugged(x)$: & $x$ is plugged. \\
            $BatteryCharged(x)$: & $x$ battery is charged. \\
            $AlarmOn(x)$: & $x$ is alarm on. \\
            $PoweredByGrid(x)$: & $x$ is the device being powered by the electrical grid. \\
            $AlarmSilenced(x)$: & $x$ if the nurse overrides the alarm manually. \\
            $On(x)$: & $x$ is the device is powered or not. \\

            \item
            Using the sets and predicates you defined express the following
            statements in predicate logic:
            \begin{enumerate}[i..]
                \item An alarm will sound whenever the line is ``pinched" or  ``plugged." \\
                $\forall p:P \bullet (Pinched(p) \lor Plugged(p)) \implies AlarmOn(p)$
                \item If there is an electrical failure the battery power will be on unless the battery is not
                properly charged. \\
                $\forall p:P \bullet (PoweredByGrid(p) \lor BatteryCharged(p)) \implies On(p)$
            \end{enumerate}
            \item Does your collection of predicates allow you to say ``The alarm
            will continue to sound until the care giver turns it off." Why or why
            not? \\
            Yes because I added the $AlarmSilenced(x)$ predicate to express if the care giver has turned the alarm off: \\
            $\forall p:P \bullet ((Pinched(p) \lor Plugged(p) \land \neg AlarmSilenced(p)) \implies AlarmOn(p)$
        \end{enumerate}
        \end{enumerate}

    \end{enumerate}


\end{document}
