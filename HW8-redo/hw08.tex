\documentclass{article}
\usepackage{zed-csp, amsfonts,color}
%%%%%%%%%%%%%%%%%%%%%%%%%%%%%%%%%%%%%%%%%%%%%%%%%%%%%%%%%%%%%%%%
%  6.826 (POCS Seminar) macro file for handouts and problem sets.
%
% You should save this file as handout.tex
%
% Your main LaTeX file should look like this:
%
%        \documentstyle[12pt]{article}
%
%        %%%%%%%%%%%%%%%%%%%%%%%%%%%%%%%%%%%%%%%%%%%%%%%%%%%%%%%%%%%%%%%%
%  6.826 (POCS Seminar) macro file for handouts and problem sets.
%
% You should save this file as handout.tex
%
% Your main LaTeX file should look like this:
%
%        \documentstyle[12pt]{article}
%
%        %%%%%%%%%%%%%%%%%%%%%%%%%%%%%%%%%%%%%%%%%%%%%%%%%%%%%%%%%%%%%%%%
%  6.826 (POCS Seminar) macro file for handouts and problem sets.
%
% You should save this file as handout.tex
%
% Your main LaTeX file should look like this:
%
%        \documentstyle[12pt]{article}
%
%        \input{handout}
%%%%%%%%%%%%%%%%%%%%%%%%%%%%%%%%%%%%%%%%%%%%%%%%%%%%%%%%%%%%%%%%

\oddsidemargin 0in
\evensidemargin 0in
\marginparwidth 40pt
\marginparsep 10pt
\topmargin 0pt
\headsep 0in
\headheight 0in
\textheight 8.5in
\textwidth 6in
\brokenpenalty=10000

% \handout{number}{date}{title}

\newcommand{\handout}[3]{


\begin{center}
\rule{\textwidth}{.0075in} \\
\rule[3mm]{\textwidth}{.0075in}\\

CMU 17-651\hfill Models of Software Systems\hfill Fall 2018\\[3ex]

{\Large\bf #3}\\[3ex]

Dario A Lencina-Talarico \hfill {\bf Handout #1} \hfill #2

\rule{\textwidth}{.0075in} \\
\rule[3mm]{\textwidth}{.0075in} \\
\end{center}

}

% \homework{number}{date}{title}{due-date}
\newcommand{\homework}[4]{

\begin{center}
\rule{\textwidth}{.0075in} \\
\rule[3mm]{\textwidth}{.0075in}\\

CMU 17-651\hfill Models of Software Systems\hfill Fall 2018\\[3ex]

{\Large\bf #3} \\[3ex]

Dario A Lencina Talarico \hfill  #1  \hfill Due: #2\\

\rule{\textwidth}{.0075in} \\
\rule[3mm]{\textwidth}{.0075in} \\
\end{center}

%\noindent
%{\bf Due date: #4}

}

% \solutionset{number}{date}{title}{due-date}
\newcommand{\solutionset}[4]{

\begin{center}
\rule{\textwidth}{.0075in} \\
\rule[3mm]{\textwidth}{.0075in}\\

CMU 17-651\hfill Models of Software Systems\hfill Fall 2016\\[3ex]

{\Large\bf #3} \\[3ex]

Garlan  \hfill  Solutions for Homework #1  \hfill  #2\\

\rule{\textwidth}{.0075in} \\
\rule[3mm]{\textwidth}{.0075in} \\
\end{center}

%\noindent
%{\bf Due date: #4}

}

% \problem{problem-number}
\newcommand{\problem}[1]{
\vspace{2ex}
\noindent
{\bf Problem #1.}

}

% \solution{solution-number}{points}
\newcommand{\solution}[2]{
\vspace{3ex}
\noindent
{\bf Problem #1}  (#2 points)

}

\newcommand{\cscomment}{
\vspace{1ex}
\noindent Comments: }

% \parts{part-alphabet}{points}
\newcommand{\parts}[2]{
\vspace{2ex}
\noindent
{\bf (#1)}  (#2 points)

}

% \problems{problems-number}{points}
\newcommand{\problems}[2]{
\vspace{3ex}
\noindent
{\bf Problem #1}  (#2 points)

}

\newenvironment{symbolfootnotes}{\def\thefootnote{\fnsymbol{footnote}}}{}

%%%%%%%%%%%%%%%%%%%%%%%%%%%%%%%%%%%%%%%%%%%%%%%%%%%%%%%%%%%%%%%%

\oddsidemargin 0in
\evensidemargin 0in
\marginparwidth 40pt
\marginparsep 10pt
\topmargin 0pt
\headsep 0in
\headheight 0in
\textheight 8.5in
\textwidth 6in
\brokenpenalty=10000

% \handout{number}{date}{title}

\newcommand{\handout}[3]{


\begin{center}
\rule{\textwidth}{.0075in} \\
\rule[3mm]{\textwidth}{.0075in}\\

CMU 17-651\hfill Models of Software Systems\hfill Fall 2018\\[3ex]

{\Large\bf #3}\\[3ex]

Dario A Lencina-Talarico \hfill {\bf Handout #1} \hfill #2

\rule{\textwidth}{.0075in} \\
\rule[3mm]{\textwidth}{.0075in} \\
\end{center}

}

% \homework{number}{date}{title}{due-date}
\newcommand{\homework}[4]{

\begin{center}
\rule{\textwidth}{.0075in} \\
\rule[3mm]{\textwidth}{.0075in}\\

CMU 17-651\hfill Models of Software Systems\hfill Fall 2018\\[3ex]

{\Large\bf #3} \\[3ex]

Dario A Lencina Talarico \hfill  #1  \hfill Due: #2\\

\rule{\textwidth}{.0075in} \\
\rule[3mm]{\textwidth}{.0075in} \\
\end{center}

%\noindent
%{\bf Due date: #4}

}

% \solutionset{number}{date}{title}{due-date}
\newcommand{\solutionset}[4]{

\begin{center}
\rule{\textwidth}{.0075in} \\
\rule[3mm]{\textwidth}{.0075in}\\

CMU 17-651\hfill Models of Software Systems\hfill Fall 2016\\[3ex]

{\Large\bf #3} \\[3ex]

Garlan  \hfill  Solutions for Homework #1  \hfill  #2\\

\rule{\textwidth}{.0075in} \\
\rule[3mm]{\textwidth}{.0075in} \\
\end{center}

%\noindent
%{\bf Due date: #4}

}

% \problem{problem-number}
\newcommand{\problem}[1]{
\vspace{2ex}
\noindent
{\bf Problem #1.}

}

% \solution{solution-number}{points}
\newcommand{\solution}[2]{
\vspace{3ex}
\noindent
{\bf Problem #1}  (#2 points)

}

\newcommand{\cscomment}{
\vspace{1ex}
\noindent Comments: }

% \parts{part-alphabet}{points}
\newcommand{\parts}[2]{
\vspace{2ex}
\noindent
{\bf (#1)}  (#2 points)

}

% \problems{problems-number}{points}
\newcommand{\problems}[2]{
\vspace{3ex}
\noindent
{\bf Problem #1}  (#2 points)

}

\newenvironment{symbolfootnotes}{\def\thefootnote{\fnsymbol{footnote}}}{}

%%%%%%%%%%%%%%%%%%%%%%%%%%%%%%%%%%%%%%%%%%%%%%%%%%%%%%%%%%%%%%%%

\oddsidemargin 0in
\evensidemargin 0in
\marginparwidth 40pt
\marginparsep 10pt
\topmargin 0pt
\headsep 0in
\headheight 0in
\textheight 8.5in
\textwidth 6in
\brokenpenalty=10000

% \handout{number}{date}{title}

\newcommand{\handout}[3]{


\begin{center}
\rule{\textwidth}{.0075in} \\
\rule[3mm]{\textwidth}{.0075in}\\

CMU 17-651\hfill Models of Software Systems\hfill Fall 2018\\[3ex]

{\Large\bf #3}\\[3ex]

Dario A Lencina-Talarico \hfill {\bf Handout #1} \hfill #2

\rule{\textwidth}{.0075in} \\
\rule[3mm]{\textwidth}{.0075in} \\
\end{center}

}

% \homework{number}{date}{title}{due-date}
\newcommand{\homework}[4]{

\begin{center}
\rule{\textwidth}{.0075in} \\
\rule[3mm]{\textwidth}{.0075in}\\

CMU 17-651\hfill Models of Software Systems\hfill Fall 2018\\[3ex]

{\Large\bf #3} \\[3ex]

Dario A Lencina Talarico \hfill  #1  \hfill Due: #2\\

\rule{\textwidth}{.0075in} \\
\rule[3mm]{\textwidth}{.0075in} \\
\end{center}

%\noindent
%{\bf Due date: #4}

}

% \solutionset{number}{date}{title}{due-date}
\newcommand{\solutionset}[4]{

\begin{center}
\rule{\textwidth}{.0075in} \\
\rule[3mm]{\textwidth}{.0075in}\\

CMU 17-651\hfill Models of Software Systems\hfill Fall 2016\\[3ex]

{\Large\bf #3} \\[3ex]

Garlan  \hfill  Solutions for Homework #1  \hfill  #2\\

\rule{\textwidth}{.0075in} \\
\rule[3mm]{\textwidth}{.0075in} \\
\end{center}

%\noindent
%{\bf Due date: #4}

}

% \problem{problem-number}
\newcommand{\problem}[1]{
\vspace{2ex}
\noindent
{\bf Problem #1.}

}

% \solution{solution-number}{points}
\newcommand{\solution}[2]{
\vspace{3ex}
\noindent
{\bf Problem #1}  (#2 points)

}

\newcommand{\cscomment}{
\vspace{1ex}
\noindent Comments: }

% \parts{part-alphabet}{points}
\newcommand{\parts}[2]{
\vspace{2ex}
\noindent
{\bf (#1)}  (#2 points)

}

% \problems{problems-number}{points}
\newcommand{\problems}[2]{
\vspace{3ex}
\noindent
{\bf Problem #1}  (#2 points)

}

\newenvironment{symbolfootnotes}{\def\thefootnote{\fnsymbol{footnote}}}{}

\newcommand{\estimates}{\mathrel{\hat{=}}}
\begin{document}

\homework{}{17 October 2018}{Homework \#8: More Z}{}


\noindent \textsc{Note}: For this assignment you must format your
answers using \LaTeX\ and typecheck your answer to question 3 using {\em fuzz},  Z-EVES, or the Community Z tools.

\begin{enumerate}

\item The following questions refer to the handout on the Library Problem.
\begin{enumerate}
\item Write an operation to see if a book is currently checked out. \\
  \color{blue}
  Assuming that we need to find if a particular copy of a book is checked out and that
  we do not need to do handle the scenario when the book is not in the library.\\
  Using the definitions of $Library$, $BookOp$, $Copy$  provided in the Lecture: \\
  \\
  Defining type to answer: \\
  Answer ::= Yes $\mid$ No 
 \begin{schema}{IsBookCheckedOut}
   \Xi BookOp \\
   book?: Copy \\
   answer!: Answer
   \where
    book? \in books \\
    (\#\{ b : books | (records\,b).status = out \} = 1) \implies answer! = Yes) \land \\
    (\#\{ b : books | (records\,b).status = out \} = 0) \implies answer! = No ) 
   
\end{schema}
\color{black}
 
 \item Suppose you are curious to find out whether other people are interested in the same books as you.
    \begin{enumerate}
    \item Is it possible to write an operation that returns the set of books that you have checked out and later returned, and that were checked out by someone else after you returned them? If so, write the operation. If not, say why. \\
      \color{blue}
      Given the definition of Data, which is the type used to store status and lastuser,
      it is not possible to preserve the history of the checkouts because every new book checkout overwrites the $lastuser$ user field. \\
      We would have to modify our records data structure to create a new record every time a
      book is Checked in or out and if someone wanted to pull the history of a book, they would have to query all records for a given book, the last known record would be the current state. \\
     \color{black} 
   \item Is it possible to write an operation that returns the set of books that you have checked out and later returned, but that were NEVER checked out by someone else after you returned them? If so, write the operation. If not, say why. \\
     \color{blue}
      Given the previous answer, it is possible, here's the operation: \\
      \begin{schema}{GetAllTheBooksThatUserHasCheckedOutAndLaterReturned}
   \Xi BookOp \\
   person?: REAL\_PERSON \\
   user\_books!:\mathbb{F}Copy
   \where
   user\_books = \{ b : books | (records\,b).lastuser = person? \land (records\,b).status = in \}
   \end{schema}
    \end{enumerate}
   \color{black} 
\end{enumerate}

\item The following questions refer to the handout on the Telephone Net.

\begin{enumerate}

\item Can a telephone Call itself? If so, what is the effect of a
  subsequent Busy operation? \\
  \\
  \color{blue}
  According to the CONNECTIONS set that was defined in the Lecture, it is possible: \\
  CONNECTIONS = $\{ \{\}, \{1\}, \{2\}, \{3\}, \{1,2\} \{1,3\}, \{2,3\}, \{1,2,3\} \}$ \\
  \\
  There's is no side effect of calling $Busy$ in the sense that the state of the Telephone Net is not changed, this is guaranteed by the invariant and the fact that part of the $Busy$ spec says that $reqs' = reqs$ \\
  \\
  A new set $\{ph?, dialled?\}$ will be added to the reqs and cons sets where $ph? = dialled?$. \\
  \color{black}

\item Write an explicit Connect operation to connect a pair of
phones. (HINT: Connect is different from Call; Connect establishes
a connection among phones on an outstanding satisfiable request. Note also that you can't use the Event framing schema here because it assumes that the starting state is an efficient net.) \\
\color{blue}
Did not try :(
\color{black}
\item Is it possible to place a call from a phone that's already
  busy?
  \color{blue}
  \\
  Lets consider the $Call$ operation defined in the lecture, $ph?, dialled? : PHONE$ are the inputs to the Z operation. \\
  \\
  The backing data structure of the TelephonyNet Schema is $\mathbb{P}$ CONNECTION \\
  \\
  Placing a call is defined as $reqs`= reqs \union \{\{ ph?,dialled? \}\} $ \\
  \\
  It is possible to place a call from a phone that is connected and it has to be made to a phone that is different than the one that we are already connected to. \\
  \\
  In order to make this a more realistic model, I propose that we do at least 2 things: \\
  1. Transitioning a request to a connection should include the possibility of a failure due to various factor such as network quality, electrical, noise etc. \\
  2. When a phone disconnects from a conference call (more than 2 peers connected) we should not drop all the participants like we do right now. \\
  \color{black}
\item Give an example that illustrates a situation in which a
Hangup implies that a new connection will be made. Does the
specification say which connection will be made if more than one
is possible?
\color{blue}
I was unable to find this scenario, probably I did not explore enough. \\
\color{black}
\item If a Hangup operation is applied to a ph? that is not yet
connected, what happens? Briefly speculate on how the
specification would have to be changed to make this more
realistic.
\color{blue}
\\
The hangoup operation is modeled as \\
$reqs' = reqs \setminus \{ c : const \vert ph? \in c \}$ \\
If $ph$ is not in $const$ then it returns an empty set, then $ reqs' = reqs \setminus \emptyset = reqs $ \\
\\
In order to make it more realistic, I propose adding a new output called success! with type $Success:== YES \vert NO$ \\
so that we can be more explicit about the success or failure of this opearation. \\
\color{black}
\item Modify Busy so that it also indicates which phones ph? is
connected to. What is the output if the input ph? is not connected
at all? \\
\color{blue}
I if not connected, dialled will return a $\emptyset$, else it will return a set with all the connections that ph has. \\
YesOrNo::= Yes $\vert$ No \\
\begin{schema}{BusyModified}
   Event \\
   ph?: PHONE \\
   activeConnections!:\mathbb{P}CONNECTION \\
   busy!:YesOrNo
   \where
   reqs' = reqs \\
   activeConnections! = \{ con : cons | ph \in con \}\\
   busy! = Yes \Leftrightarrow ph? \in \union cons 
   \end{schema}
\color{black}

\end{enumerate}

\item The following scenario describes a typical classroom
situation.

\begin{enumerate} \item A teacher needs to keep track of
which homework assignments each student in the class has turned
in. Each student in the class is given an ID, and at any time each
student in the class has a (possibly empty) set of homeworks that
have been turned in. The system should only keep records of the
students in {\em this\/} class.

\begin{zed}
    [ID,STUDENTNAME,HOMEWORK]
\end{zed}
%%Unchecked
\begin{schema}{ClassRecords}
    student:ID \pfun STUDENTNAME \\
    turnedIn: STUDENTNAME \pfun \power HOMEWORK
\where
\color{blue}
dom\,turnedIn \subseteq range\,student 
\end{schema}

Complete the schema with an appropriate invariant.

\textbf{Note:} You are not allowed to change the state variables
in the schema. Supply only an appropriate invariant.

%\vspace{12pt}

% Problem 2
\item Which of the following can be inferred from your definition
of $ClassRecords$? Briefly justify your answer.

\begin{itemize}
\item No two students are assigned the same $ID$.
  %\vspace{.65in}
  \\
  \textcolor{blue}{This can be inferred since the definition of student as $ID \pfun STUDENTNAME$ guarantees that.\\
\\    
    Meaning, it is not possible to map the same id to two different students.
  }

\item A given student may have more than one $ID$.
  %\vspace{.65in}
  \textcolor{blue}{The proposed definition does allow that if the enrollment people happen to use a variation of the student's name for each id.}

\item There may be some $ID$s that are not used by the system.
  %\vspace{.65in}
  \textcolor{blue}{To the best of my knowledge, the assignement does not provide a very specific definition of the $ID$ type or domain, I think it is safe to say that there will be many ids that wont be used by the system.}

\item All students that have an $ID$ also have a set of
  $HOMEWORK$s.
  \textcolor{blue}{The current definition of the system does not provide this guarantee because the turnedIn partial function is not initialized}
%\vspace{.65in}

\item Any student who has a set of $HOMEWORK$s also has an $ID$.
%\vspace{.65in}
 \textcolor{blue}{Yes, the invariant that I introduced focuses on this property by defining that the domain of turnedIn has to be in or equal the range of student, that implies that the sutdentname has to be associated with an id.}
\end{itemize}


% Problem 3
\item Write a schema $InitClassRecords$ that defines an
appropriate initial state space for the system. Explain why the
initial state space is consistent with the state space invariant
that you defined earlier.
%\vspace{2in}
\color{blue}
I could not find how to init partial functions, is the intention to enroll all the students here? \\
If that is not the case then an Empty Initializer will respect the invariant as both the $stuent$ and $turnedIn$ partial functions are empty, meaning, no range and no domain.
\begin{schema}{InitClassRecords}
   ClassRecords
\where
\color{blue}
\end{schema}
\color{black}
% Problem 4
\item Write an operation to add a student to the class, provided
that the student is not already a member of the class.
%\vspace{2.5in}
\color{blue}
\begin{schema}{AddStudent}
  \Delta ClassRecords \\
  id?:ID \\
  student\_name?:STUDENTNAME
\where
id \notin dom\, student \\
student' = student \union \{id? \mapsto student\_name?\} \land turnedIn' = turnedIn
\end{schema}
\color{black}

% Problem 5
\item Write a robust version of the operation that returns an error
value if the student is already a member of the class. Use the schema
calculus: you should not need to rewrite the original operation.
%\vspace{3in}
\color{blue}\\
Declaring return type to encapsulate success/error. \\
$ADD\_STUDENT\_RESULT$ ::= success $\mid student\_already\_enrolled$ \\
\begin{schema}{AddStudentSuccess}
    result!:$ADD\_STUDENT\_RESULT$
  \where
    result!: success
\end{schema}

\begin{schema}{AddStudentError}
    result!:$ADD\_STUDENT\_RESULT$ 
    \where
    id \in dom\, student \\ 
    result!: student\_already\_enrolled
\end{schema}

With the defined schemas we can proceed to create the robust AddStudent: \\
$RAddStudent \estimates AddStudent \land AddStudentSuccess \lor AddStudentError$ \\

\color{black}
% Problem 6
\item Write an operation, $DeadBeats$, that returns the set of $ID$s
(not student names!) of students who have not turned in more than one homework assignment.
%\vspace{2in}
  \color{blue}
\begin{schema}{DeadBeats}
  \Xi ClassRecords \\
  dead\_beats!: \mathbb{P} ID
  \where
  \forall id : (dom student) \bullet \\
     \#\{turnedIn(student(id)\} < 2 \land dead\_beats = dead\_beats \union {id}, student' = student \land student' = student 
\end{schema}
\color{black}

% Problem 7
\item Consider the following globally-defined function that
determines whether a student's status is ok or not, based on a
comparison with the set of total assignments that could have been
turned in.

\begin{zed}
    STATUS ::= ok | not\_ok
\end{zed}

\begin{axdef}
    StatusOf: \power HOMEWORK \cross \power HOMEWORK \fun STATUS
\where
    \forall student, total: \power HOMEWORK @ \\
\t1    (StatusOf(student,total) = ok) \iff (\#(total \setminus
student) < 2)
\end{axdef}

Explain in informal terms when the status of a student is not ok.
%\vspace{1in}

\textcolor{blue}{
  I know that this is a recursive function but I did not understand when the
  iteration stop, maybe when $\#(total \ student ) < 2$ ?
  What does the $X$ operator mean? 
}

\end{enumerate}


\end{enumerate}

\end{document}
