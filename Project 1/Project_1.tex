\documentclass{article}
\usepackage{times}
\usepackage{fullpage}

\newcommand{\head}{\subsection*}
\setlength{\parindent}{0pt}

\begin{document}

%_______________________________________________________________________________
% Heading section
%_______________________________________________________________________________
\begin{center}
\rule{\textwidth}{1.5pt} \\ \rule[10pt]{\textwidth}{1pt}\\
CMU 17-651\hfill Models of Software Systems\\[3ex]
{\Large\bf Project 1: State Machine Modeling}\\[3ex]
Garlan \hfill {\bf Due: October 5, 2016} \rule{\textwidth}{1pt}
\\\rule[9.5pt]{\textwidth}{1.5pt}
\end{center}

The purpose of this first project is to give you experience in
modeling a realistic system as a state machine. The example that we
will use is the Infusion Pump. A general description of an Infusion
Pump can be found in the General Project Documents folder on
Blackboard.

\bigskip You should carry out this project in your assigned team. Make sure that everyone in the group contributes to the overall effort. Each team should submit a single write-up of the project,
due at the beginning of class on the project due date. We have
posted a template for a group project write-up under the Course
Resources $>$ \LaTeX~section on Blackboard.

%___________________________________________________________________________________________________
\head{Task 1 (20 points):}
%___________________________________________________________________________________________________

A sample description of a \emph{simplified version} of an infusion
pump, written in FSP, is provided with this project document. This
model describes a pump with only one infusion channel, and leaves
out many of the features that a real infusion pump would have, as
outlined in the general infusion pump description.

\bigskip Your first task is to understand this specification. Read
through the specification to make sure you understand what it is
specifying. Then read it into LTSA and check its behavior using the
LTSA simulation capabilities. Once you are familiar with it, answer
the following questions in your project write-up:

\begin{enumerate}
    \item What is the alphabet of the state machine? \\
      $\{change\_settings, clear\_rate, confirm\_settings, connect\_set, dispense\_main\_med\_flow, enter\_value,$ \\
      $erase\_and\_unlock\_line, flow\_blocked, flow\_unblocked, lock\_line, lock\_unit, plug\_in, press\_cancel,$\\
      $press\_set, purge\_air, set\_rate, silence\_alarm, sound\_alarm, turn\_off, turn\_on, unlock\_unit, unplug \}$

    \item List two traces of the pump, each at least 4 actions in length. \\
$\langle plug\_in, turn\_on, set\_rate, enter\_value \rangle$ \\
$\langle plug\_in, turn\_on, turn\_off, un\_plug \rangle$
    \item In contrast to the specification of an infusion pump in homework 6, how does this specification model the fact that     the pump might run out of liquid? \\
      \\
      This specification models the fact that the pump might run out of liquid the same way it models the blocked line. In either case the system will raise the alarm.

    \item Is it possible to ever dispense medication without setting the rate?  Why or why not? If your answer is yes, provide a trace that justifies your answer. \\
No, it is not possible to dispense medicine without setting the rate. This is because once the user turns on the pump using the action $turn\_on$, they only have the options to either unplug, $turn\_off$ or $set\_rate$. So in order to move ahead and start the infusion, they must set the rate using the action $set\_rate$. \\
A trace depicting this:\\
$\langle plug\_in, turn\_on, set\_rate, enter\_value, press\_set, connect\_set, purge\_air, lock\_line, confirm\_settings,$\\ $lock\_unit, dispense\_main\_med\_flow, dispense\_main\_med\_flow, ... \rangle$
      
\item Is it ever possible for the flow to become blocked and have the alarm not sound at all?  Why or why not? If your answer is yes, provide a trace that justifies your answer. \\
No, it is not possible for the flow to become blocked and have the alarm not sound at all. This is because once the flow is blocked using the action $flow\_blocked$, there is only one possible action which is sound alarm. \\
A trace depicting this:\\
$\langle plug\_in, turn\_on, set\_rate, enter\_value, press\_set, connect\_set, purge\_air, lock\_line, confirm \_settings,$\\ $lock\_unit, dispense\_main\_med\_flow, flow\_blocked, sound\_alarm \rangle$
  
  
\item If the pump is locked and dispensing, without unlocking or becoming blocked, will the pump ever stop dispensing? If your answer is yes, provide a trace that justifies your answer. \\
Yes, the infusion could stop dispensing once it is locked and dispensing if it is turned off or unplugged. However, it is not turned off or unplugged, it would could dispensing.\\
The following is a trace depicting it being turned off:\\

$\langle plug\_in, turn\_on, set\_rate, enter\_value, press\_set, connect\_set, purge\_air, lock\_line, confirm\_settings,$\\
$lock\_unit, turn\_off \rangle$
  
\item If the pump is locked and dispensing, is it possible for the \emph{patient} to alter the medicine he is receiving? If your answer is yes, provide a trace that justifies your answer.\\
The answer is yes. The following is a trace depicting: \\
$\langle plug\_in, turn\_on, set\_rate, enter\_value, press\_set, connect\_set, purge\_air, lock\_line, confirm\_settings,$\\ $lock\_unit, dispense\_main\_med\_flow, unlock\_unit, change\_settings, confirm\_settings, dispense\_main\_med\_flow \rangle$
  
\item Does this version have any behavior that you feel is inconsistent with the pump specification? Could it be fixed? \\  
One inconsistency could be the fact that the only way to stop the infusion pump is to turn off the pump, which is inconsistent with the intended design of the pump.
\\
If we were to add a Hold function that could pause the infusion, and then the pump could move to a Hold state.\\

\end{enumerate}

%___________________________________________________________________________________________________
\head{Task 2 (75 Points):}
%___________________________________________________________________________________________________

For the second part of your project you should develop a
more-complete FSP specification of the infusion pump (40 points). To
do this you can use the sample of Task 1 as a starting point, or you
can start with your own model. Here are some guidelines to keep in
mind as you develop your FSP specification:

\begin{itemize}
\item You do not need to model the human user of the pump.

\item Restrict your specification to a single-line infusion pump.

\item You are free to pick the level of abstraction for this specification. Your specification
should be detailed enough, however, to answer the questions posed below.

\item Be sure to document your specification adequately, and choose meaningful action and process names for readability.

\item Your specification should not use parallel composition.

\end{itemize}

\noindent The full specification should be attached to your project
write-up. Answer the following questions (35 points); for each,
briefly explain why you answered it in the way you did, based on
\emph{your} model. If your model does not address this issue,
explain why. (Note: Reference certain parts of your specification that address features, ambiguities, or errors in each question.)

\begin{enumerate}
\item Which aspects of the pump did you choose to model, and which did you choose to leave out? \\
  We took the existing fsp model as a start point and added the following features: \\
  \begin{enumerate}
  \item Added hold button.
  \item Added better error handling: in particular, distinguish between blocked line and out of liquid.
    \item TODO: add more.
  \end{enumerate}
    \item Were there ambiguities in the English description of the infusion pump that your specification resolves?
    \item State four general properties that your pump guarantees (for example, the alarm will always sound if a line becomes clogged), and say briefly why it is guaranteed.
    \item Does your model say what happens if the power goes out in the middle of operation?
    \item Referring to the additional documentation about the infusion pump on the general project description section of the web site, consider the errors noted about realistic pumps. Which of these types of errors can be illustrated with your pump? Does your pump exclude some of them from happening? \\
      \begin{enumerate}
      \item FDA Recall: The machines, made by Baxter International and known as Colleague volumetric infusion pumps, shut down for a variety of software, wiring and design reasons. One of the most basic problem is that the ``On/Off'' key is so close to the ``Start'' key that nurses may inadvertently turn the machines off when they actually intend to begin drug theray, the agency said.
      \item FDA Recall retrieved from ($https://www.tennesseelawblog.com/fda\_recalls\_medtronic\_infusion$) a defective pump motor has been known to stall, which means the delivery of vital drugs will suddenly stop without notice.
      \item FDA Recall retrieved from ($https://www.youhavealawyer.com/blog/2008/01/12/infusion-pump-recall/$) The hospital infusion pump recall was issued because the Alaris pumps could contain misassembled occluder springs, which is used to control the flow of medication. The springs could be bent, broken, nested or missing, which may result in too much medication or fluid being delivered. The FDA has classified the action as a Class 1 recall, since use of the defective Alaris/Medley infusion pumps involve a reasonable probability of serious injury or death.\\
\\
Overinfusion is difficult for the hospital or medical provider to detect. The defective infusion pump springs could work intermittently, and there is no warning or alert that the device is delivering too much of the fluid. 
      \end{enumerate}

\end{enumerate}


%___________________________________________________________________________________________________
\head{Task 3 (5 points):}
%___________________________________________________________________________________________________


How difficult would it be using the current subset of FSP to create
a 4-line pump? What would have to change in your specification?





\end{document}
