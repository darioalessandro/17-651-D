\documentclass{article}
\usepackage{graphics,indentfirst,amsmath,amsthm,amssymb,latexsym,enumerate}
\usepackage{graphicx}
\usepackage{zed-csp}
\input{handout}

\begin{document}

\homework{}{26 September 2016}{Homework \#5: State Machines}{}

\begin{enumerate}

\item A certain, simple, answering machine has two buttons, ``play''
and ``save'' and can, of course, receive messages. If someone plays
the messages and doesn't save them, they are erased/overwritten when
the next incoming message is received. The answering machine only
holds a specified number of messages; when it reaches full capacity
it refuses to accept new messages. The answering machine can be
modeled by the state machine, $\mathbf{AnsMachine}$, whose state
transition diagram is attached.

\includegraphics[width=4in]{ansmachine.png}

\begin{enumerate}
 \item Give a 4-tuple description for this state machine.
 \item Give three execution fragments of {\bf AnsMachine}, at least one of which is not an
 execution.
 \item Give both a finite and an infinite execution of
 {\bf AnsMachine}.
 \item For each of the following, indicate whether or not it is an event-based trace of
 {\bf AnsMachine}.
 \begin{enumerate}
  \item  $\langle play, save, msg, msg, play \rangle$
  \item  $\langle msg, msg, msg, play, save, msg, msg, msg \rangle$
  \item  $\langle msg, play, save, msg, msg, play, msg, msg, msg \rangle$
 \end{enumerate}
 \item Give two examples of state-based traces of
 {\bf AnsMachine}.
 \item Give two sequences of
states that are not state-based traces of {\bf AnsMachine}.
\item What are the reachable states of this machine?
\item Is {\bf AnsMachine}'s event-based behavior finite or
infinite?
 \item Can a state machine $M=(S,I,A,\delta)$ with an infinite trace have finite
behavior? Give an example or explain why not.
\end{enumerate}

\item Consider a TV remote control that allows the user to select
the channel to be viewed, add or remove a ``parental block'' to a
channel that prevents the channel from being displayed (removal
requires a password), and enter a password to allow a blocked
channel currently selected to be displayed. If a blocked channel
is selected, the channel is not initially displayed. The user may
choose to select a different channel or may enter the password to
display the channel. If the incorrect password is entered, the
channel is not displayed. If the correct password is entered, the
channel is displayed.

Your task is to model the described functionality of the remote control. That is,
\begin{enumerate}
\item Specify the set of states

\item Specify the pre- and post-conditions for each action
\end{enumerate}

Your solution should satisfy the following requirements:
\begin{enumerate}[i.]
\item Do \emph{not} model any functionality other than that described above. In particular, assume that the correct password is fixed and cannot be changed.
\item Assume that the set of channels is $Channels == \set{n:\nat \mid 1 \leq n \leq 100}$.
\item You may only use the following actions in your model:
    \begin{itemize}
    \item $select$: Select a channel for viewing
    \item $correctpw$: Enter correct password
    \item $incorrectpw$: Enter incorrect password
    \item $addblock$: Block selected channel
    \item $removeblock$: Unblock selected channel
    \end{itemize}
\item If (and only if) the requirements are ambiguous, state any assumptions that you made to resolve those ambiguities.
\end{enumerate}


\end{enumerate}

\end{document}
