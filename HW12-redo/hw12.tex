\documentclass{article}
\usepackage{amsfonts,enumerate,zed-csp}
\input{handout}
\newcommand{\Until}{\,\mathcal{U}\,}
\newcommand{\Next}{\bigcirc}
\usepackage{graphicx}
\begin{document}

\homework{}{14 November 2016}{Homework \#12: LTL and FSP Concurrency}{}

\begin{enumerate}
\item For each of the following pairs, either argue (informally) why they are equivalent, or provide a counterexample trace that shows they are not equivalent.

    {\sc Note} regarding counterexamples: To show that, for example, $\Box (p \land q)$ and $\Box (p \lor q)$ are {\em not} equivalent we provide the counterexample trace $\langle (p,\neg q), (p,\neg q), \ldots \rangle$. This trace is read as follows: ``in state 1 $p$ is true and $q$ is false, in state 2 $p$ is true and $q$ is false, and so on for the entire trace." $\Box (p \lor q)$ is true for the given trace since $p$ is true in every state of the trace (and hence so is $p \lor q$), but $\Box (p \land q)$ is not true since that would require both $p$ and $q$ to be true in every state of the trace.
\begin{enumerate}
\item \makebox[1.5in][l]{$\Diamond p\land\Diamond q$}
  $\Diamond(p\land\Diamond q)\lor\Diamond(q\land\Diamond p)$
  \\ equivalent: \\ In the original assignement I classified it as not
  equivalent because I failed to understand the meaning of the outer
  $\Diamond$ operator in $\Diamond(q\land\Diamond p)$. I said that if
  q was false in the first state then the whole thing would become
  false which is not true. if q is true in any state and p is true in the same state or afterwards, then that expression would be true. \\
\item \makebox[1.5in][l]{$\Diamond p\land\Diamond q$}  $\Diamond(p\land q)$ \\
  not equivalent.\\
  sample trace = $\langle (\neg p, q), (p, \neg q) \rangle$ is true for the expression on the left but not for the one on the right as p and q are not true at the same time in any state. \\
\item \makebox[1.5in][l]{$\Box(p\lor q)$} $\Box p \lor \Box q$ \\
  not equivalent \\
  sample trace = $\langle (\neg p, q), (p, \neg q) \rangle$ \\
  is true for the expression on the left side but not for the one of the right as neither p or q are true forever. \\
\item \makebox[1.5in][l]{$(p\land q)\Until r$} $(p\Until r)\land (q\Until r)$ \\
 equivalent.
  
\end{enumerate}
\item Assuming that the following are true of $\sigma$:
\begin{itemize}
\item $\Box((p\implies q)\lor s)$
\item $(\sigma,3)\models\Box p $
\item $(\sigma,3)\models\Next(q\land\Next\Box r) $
\item $(\sigma,4)\models\Box(r\implies\neg q) $
\end{itemize}
which of the following are true, which are false, and which could be either?
\begin{enumerate}
\item $(\sigma,5)\models q$ \\
  $(\sigma,3)\models\Next(q\land\Next\Box r)$  tells us that when $\sigma = 5$ then $\Box r $ will be true, \\
  then because $(\sigma,4)\models\Box(r\implies\neg q) $ then q has to be false. 
 \item $(\sigma,4)\models s$ \\ s could be true or false because $p \implies q$ is true which satisfies $\Box((p\implies q)\lor s)$ \\
 \item $(\sigma,5)\models s$ \\
   s has to be true because q is false, so s needs to satisfy  $\Box((p\implies q)\lor s)$\\
\item $(\sigma,3)\models q\lor s$ \\
  true because $(\sigma,3)\models\Box p $ and  $\Box((p\implies q)\lor s)$ \\
\item $(\sigma,4)\models r$ \\
  r has to be false because q is true due to $(\sigma,3)\models\Next(q\land\Next\Box r) $ \\
\end{enumerate}


\item The following is an FSP model of the alternating-bit communication protocol
over an unreliable link:

\begin{small}
\begin{verbatim}
const Max = 1 range Msg = 0..Max

SENDER = SENDER[0], SENDER[i:Msg] = (
    send_msg[i] -> (
        msg_received[i] -> (
            // proceed to send the next message
            ack_received[i] -> SENDER[(i + 1) % (Max + 1)]
            |
            // retransmit
            ack_timeout[i] -> SENDER[i]
        )
        |
        // retransmit
        msg_timeout[i] -> SENDER[i]
    )
).

RECEIVER = RECEIVER[0][1], RECEIVER[i:Msg][j:Msg] = (
    msg_received[i] -> send_ack[i] -> ( // process the currently expected message
        ack_received[i] -> RECEIVER[(i + 1) % (Max + 1)][i]
        |
        ack_timeout[i] -> RECEIVER[(i + 1) % (Max + 1)][i]
    )
    |
    msg_received[j] -> send_ack[j] -> ( // process the previously expected message
        ack_received[j] -> RECEIVER[(j + 1) % (Max + 1)][j]
        |
        ack_timeout[j] -> RECEIVER[(j + 1) % (Max + 1)][j]
    )
).

||ALTBITPROTOCOL = (SENDER || RECEIVER).
\end{verbatim}
\end{small}

As shown by the model, this protocol follows the ``stop-and-wait''
style. That is, a new message is not transmitted from the sender to
the receiver unless (1)~the receiver has sent back an acknowledgment
and (2)~the sender has received that acknowledgement. Since, the
link is unreliable, both messages and acknowledgments may be lost at
any time. Also, notice that the link is ``half-duplex''---meaning
that transmissions go over one direction at a time.

\begin{enumerate}
\item Use LTSA to check if this protocol is deadlock free. Briefly explain why the protocol is deadlock free or why it is not. \\
\includegraphics[scale=0.5]{deadlocks.png}
According to the checker, the protocol is deadlock free as eventually both messages will be received and acknowledged correctly even when it takes many attempts due to the msg and ack timeouts.

\item Define fluents \verb"MSG_SENT" and \verb"ACK_SENT" and an FLTL
formula which uses those fluents and states that every message
transmitted by the sender is eventually retrieved by the receiver.

\begin{verbatim}
fluent MSG_SENT[i:Msg] = <send_msg[i], msg_received[i]>

fluent ACK_SENT[i:Msg] = <send_ack[i], ack_received[i]>

assert ALL_MESSAGES_ARE_RECEIVED = forall[i:Msg] [] (MSG_SENT[i] -> <>ACK_SENT[i])
\end{verbatim}

\item Use LTSA to check if the protocol satisfies your LTL property. Briefly explain why the property is satisfied or why it is not.

\begin{verbatim}
Composition:
ALTBITPROTOCOL = SENDER || RECEIVER || ALL_MESSAGES_ARE_RECEIVED
State Space:
 6 * 10 * 12 = 2 ** 11
LTL Property Check...
-- States: 59 Transitions: 108 Memory used: 240200K
No LTL Property violations detected.
LTL Property Check in: 0ms
\end{verbatim}

According to LTL, the property is satisfied, I think that the reason is that even when timeouts occur, because of the $\Diamond$ that I used, eventually the message will make it to the receiver and the ack will be received by the sender.

\item Modify the model by removing the \verb"*_timeout" transition
choices and rerun LTSA to check if the modified protocol is deadlock
free. Briefly explain why the modified protocol is deadlock free or why it is not.
\begin{verbatim}
const Max = 1 range Msg = 0..Max

SENDER = SENDER[0], SENDER[i:Msg] = (
    send_msg[i] -> (
        msg_received[i] -> (
            // proceed to send the next message
            ack_received[i] -> SENDER[(i + 1) % (Max + 1)]
        )
    )
).

RECEIVER = RECEIVER[0][1], RECEIVER[i:Msg][j:Msg] = (
    msg_received[i] -> send_ack[i] -> ( // process the currently expected message
        ack_received[i] -> RECEIVER[(i + 1) % (Max + 1)][i]
    )
    |
    msg_received[j] -> send_ack[j] -> ( // process the previously expected message
        ack_received[j] -> RECEIVER[(j + 1) % (Max + 1)][j]
    )
).

||ALTBITPROTOCOL = (SENDER || RECEIVER).

fluent MSG_SENT[i:Msg] = <send_msg[i], msg_received[i]>

fluent ACK_SENT[i:Msg] = <send_ack[i], ack_received[i]>

assert ALL_MESSAGES_ARE_RECEIVED = forall[i:Msg] [] (MSG_SENT[i] -> <>ACK_SENT[i])

\end{verbatim}
\includegraphics[scale=0.5]{nodeadlocks2.png}

According to the ltsa checker, the system is deadlock free as there is no way for the receiver to block the sender or the other way around, the second of the system imo is less desirable as timeouts are a useful and realistic error scenario that makes the model easier to implement. \\
\end{enumerate}
NOTE: For every question in which you are asked to use the LTSA LTL
property checker, you need to include the actual resulting output of
the checker.

\end{enumerate}


\end{document}
