\documentclass{article}
\usepackage{verbatim,indentfirst,amsmath,amsthm,amssymb,latexsym,enumerate,amssymb,zed-csp}
\usepackage[usenames,dvipsnames]{xcolor}

\newcommand{\revise}[1] {\textcolor{red} {#1}} 				% Text to revise


%%%%%%%%%%%%%%%%%%%%%%%%%%%%%%%%%%%%%%%%%%%%%%%%%%%%%%%%%%%%%%%%
%  6.826 (POCS Seminar) macro file for handouts and problem sets.
%
% You should save this file as handout.tex
%
% Your main LaTeX file should look like this:
%
%        \documentstyle[12pt]{article}
%
%        %%%%%%%%%%%%%%%%%%%%%%%%%%%%%%%%%%%%%%%%%%%%%%%%%%%%%%%%%%%%%%%%
%  6.826 (POCS Seminar) macro file for handouts and problem sets.
%
% You should save this file as handout.tex
%
% Your main LaTeX file should look like this:
%
%        \documentstyle[12pt]{article}
%
%        %%%%%%%%%%%%%%%%%%%%%%%%%%%%%%%%%%%%%%%%%%%%%%%%%%%%%%%%%%%%%%%%
%  6.826 (POCS Seminar) macro file for handouts and problem sets.
%
% You should save this file as handout.tex
%
% Your main LaTeX file should look like this:
%
%        \documentstyle[12pt]{article}
%
%        \input{handout}
%%%%%%%%%%%%%%%%%%%%%%%%%%%%%%%%%%%%%%%%%%%%%%%%%%%%%%%%%%%%%%%%

\oddsidemargin 0in
\evensidemargin 0in
\marginparwidth 40pt
\marginparsep 10pt
\topmargin 0pt
\headsep 0in
\headheight 0in
\textheight 8.5in
\textwidth 6in
\brokenpenalty=10000

% \handout{number}{date}{title}

\newcommand{\handout}[3]{


\begin{center}
\rule{\textwidth}{.0075in} \\
\rule[3mm]{\textwidth}{.0075in}\\

CMU 17-651\hfill Models of Software Systems\hfill Fall 2018\\[3ex]

{\Large\bf #3}\\[3ex]

Dario A Lencina-Talarico \hfill {\bf Handout #1} \hfill #2

\rule{\textwidth}{.0075in} \\
\rule[3mm]{\textwidth}{.0075in} \\
\end{center}

}

% \homework{number}{date}{title}{due-date}
\newcommand{\homework}[4]{

\begin{center}
\rule{\textwidth}{.0075in} \\
\rule[3mm]{\textwidth}{.0075in}\\

CMU 17-651\hfill Models of Software Systems\hfill Fall 2018\\[3ex]

{\Large\bf #3} \\[3ex]

Dario A Lencina Talarico \hfill  #1  \hfill Due: #2\\

\rule{\textwidth}{.0075in} \\
\rule[3mm]{\textwidth}{.0075in} \\
\end{center}

%\noindent
%{\bf Due date: #4}

}

% \solutionset{number}{date}{title}{due-date}
\newcommand{\solutionset}[4]{

\begin{center}
\rule{\textwidth}{.0075in} \\
\rule[3mm]{\textwidth}{.0075in}\\

CMU 17-651\hfill Models of Software Systems\hfill Fall 2016\\[3ex]

{\Large\bf #3} \\[3ex]

Garlan  \hfill  Solutions for Homework #1  \hfill  #2\\

\rule{\textwidth}{.0075in} \\
\rule[3mm]{\textwidth}{.0075in} \\
\end{center}

%\noindent
%{\bf Due date: #4}

}

% \problem{problem-number}
\newcommand{\problem}[1]{
\vspace{2ex}
\noindent
{\bf Problem #1.}

}

% \solution{solution-number}{points}
\newcommand{\solution}[2]{
\vspace{3ex}
\noindent
{\bf Problem #1}  (#2 points)

}

\newcommand{\cscomment}{
\vspace{1ex}
\noindent Comments: }

% \parts{part-alphabet}{points}
\newcommand{\parts}[2]{
\vspace{2ex}
\noindent
{\bf (#1)}  (#2 points)

}

% \problems{problems-number}{points}
\newcommand{\problems}[2]{
\vspace{3ex}
\noindent
{\bf Problem #1}  (#2 points)

}

\newenvironment{symbolfootnotes}{\def\thefootnote{\fnsymbol{footnote}}}{}

%%%%%%%%%%%%%%%%%%%%%%%%%%%%%%%%%%%%%%%%%%%%%%%%%%%%%%%%%%%%%%%%

\oddsidemargin 0in
\evensidemargin 0in
\marginparwidth 40pt
\marginparsep 10pt
\topmargin 0pt
\headsep 0in
\headheight 0in
\textheight 8.5in
\textwidth 6in
\brokenpenalty=10000

% \handout{number}{date}{title}

\newcommand{\handout}[3]{


\begin{center}
\rule{\textwidth}{.0075in} \\
\rule[3mm]{\textwidth}{.0075in}\\

CMU 17-651\hfill Models of Software Systems\hfill Fall 2018\\[3ex]

{\Large\bf #3}\\[3ex]

Dario A Lencina-Talarico \hfill {\bf Handout #1} \hfill #2

\rule{\textwidth}{.0075in} \\
\rule[3mm]{\textwidth}{.0075in} \\
\end{center}

}

% \homework{number}{date}{title}{due-date}
\newcommand{\homework}[4]{

\begin{center}
\rule{\textwidth}{.0075in} \\
\rule[3mm]{\textwidth}{.0075in}\\

CMU 17-651\hfill Models of Software Systems\hfill Fall 2018\\[3ex]

{\Large\bf #3} \\[3ex]

Dario A Lencina Talarico \hfill  #1  \hfill Due: #2\\

\rule{\textwidth}{.0075in} \\
\rule[3mm]{\textwidth}{.0075in} \\
\end{center}

%\noindent
%{\bf Due date: #4}

}

% \solutionset{number}{date}{title}{due-date}
\newcommand{\solutionset}[4]{

\begin{center}
\rule{\textwidth}{.0075in} \\
\rule[3mm]{\textwidth}{.0075in}\\

CMU 17-651\hfill Models of Software Systems\hfill Fall 2016\\[3ex]

{\Large\bf #3} \\[3ex]

Garlan  \hfill  Solutions for Homework #1  \hfill  #2\\

\rule{\textwidth}{.0075in} \\
\rule[3mm]{\textwidth}{.0075in} \\
\end{center}

%\noindent
%{\bf Due date: #4}

}

% \problem{problem-number}
\newcommand{\problem}[1]{
\vspace{2ex}
\noindent
{\bf Problem #1.}

}

% \solution{solution-number}{points}
\newcommand{\solution}[2]{
\vspace{3ex}
\noindent
{\bf Problem #1}  (#2 points)

}

\newcommand{\cscomment}{
\vspace{1ex}
\noindent Comments: }

% \parts{part-alphabet}{points}
\newcommand{\parts}[2]{
\vspace{2ex}
\noindent
{\bf (#1)}  (#2 points)

}

% \problems{problems-number}{points}
\newcommand{\problems}[2]{
\vspace{3ex}
\noindent
{\bf Problem #1}  (#2 points)

}

\newenvironment{symbolfootnotes}{\def\thefootnote{\fnsymbol{footnote}}}{}

%%%%%%%%%%%%%%%%%%%%%%%%%%%%%%%%%%%%%%%%%%%%%%%%%%%%%%%%%%%%%%%%

\oddsidemargin 0in
\evensidemargin 0in
\marginparwidth 40pt
\marginparsep 10pt
\topmargin 0pt
\headsep 0in
\headheight 0in
\textheight 8.5in
\textwidth 6in
\brokenpenalty=10000

% \handout{number}{date}{title}

\newcommand{\handout}[3]{


\begin{center}
\rule{\textwidth}{.0075in} \\
\rule[3mm]{\textwidth}{.0075in}\\

CMU 17-651\hfill Models of Software Systems\hfill Fall 2018\\[3ex]

{\Large\bf #3}\\[3ex]

Dario A Lencina-Talarico \hfill {\bf Handout #1} \hfill #2

\rule{\textwidth}{.0075in} \\
\rule[3mm]{\textwidth}{.0075in} \\
\end{center}

}

% \homework{number}{date}{title}{due-date}
\newcommand{\homework}[4]{

\begin{center}
\rule{\textwidth}{.0075in} \\
\rule[3mm]{\textwidth}{.0075in}\\

CMU 17-651\hfill Models of Software Systems\hfill Fall 2018\\[3ex]

{\Large\bf #3} \\[3ex]

Dario A Lencina Talarico \hfill  #1  \hfill Due: #2\\

\rule{\textwidth}{.0075in} \\
\rule[3mm]{\textwidth}{.0075in} \\
\end{center}

%\noindent
%{\bf Due date: #4}

}

% \solutionset{number}{date}{title}{due-date}
\newcommand{\solutionset}[4]{

\begin{center}
\rule{\textwidth}{.0075in} \\
\rule[3mm]{\textwidth}{.0075in}\\

CMU 17-651\hfill Models of Software Systems\hfill Fall 2016\\[3ex]

{\Large\bf #3} \\[3ex]

Garlan  \hfill  Solutions for Homework #1  \hfill  #2\\

\rule{\textwidth}{.0075in} \\
\rule[3mm]{\textwidth}{.0075in} \\
\end{center}

%\noindent
%{\bf Due date: #4}

}

% \problem{problem-number}
\newcommand{\problem}[1]{
\vspace{2ex}
\noindent
{\bf Problem #1.}

}

% \solution{solution-number}{points}
\newcommand{\solution}[2]{
\vspace{3ex}
\noindent
{\bf Problem #1}  (#2 points)

}

\newcommand{\cscomment}{
\vspace{1ex}
\noindent Comments: }

% \parts{part-alphabet}{points}
\newcommand{\parts}[2]{
\vspace{2ex}
\noindent
{\bf (#1)}  (#2 points)

}

% \problems{problems-number}{points}
\newcommand{\problems}[2]{
\vspace{3ex}
\noindent
{\bf Problem #1}  (#2 points)

}

\newenvironment{symbolfootnotes}{\def\thefootnote{\fnsymbol{footnote}}}{}
\renewcommand{\implies}{\Rightarrow}
\renewcommand{\iff}{\Leftrightarrow}
\newcommand{\define}{==}
\begin{document}

\homework{}{24 September 2018}{Homework \#4: \\[10pt]
Structures, Equational Reasoning, and Induction }{}

\begin{enumerate}[\bf I.]
\item \textbf{Sets, Relations, and Functions}
\begin{enumerate}[1.]
\item Use set comprehension to define the set $\mathit{SumOfSquares}$
containing all the natural numbers that can be expressed as the sum
$a^2 + b^2$ where $a$ and $b$ are natural numbers. \\
\textbf{Redo note:} Adjusted parenthesis \\
$SumOfSquares == \{x: \mathbb{N} | \exists a : \mathbb{N} \bullet \exists b : \mathbb{N} \bullet  x = a^2 + b^2 \}$ \\

\item Write out in full the powersets of each of the following.
\begin{enumerate}[a.]
\item $\mathbb{P}\{7,1\} = \{\emptyset, \{1\}, \{7\}, \{7,1\}\}$
\item $\mathbb{P}\{5\}$ = $\{\emptyset, \{5\}\}$
\item $\mathbb{P} \emptyset$ = $\{\emptyset\}$
\item $\mathbb{P}\{\emptyset\}$ = $\{\emptyset, \{\emptyset\}\}$
\end{enumerate}
\item Write out in full the following Cartesian products.
\begin{enumerate}[a.]
\item $\{4,2\}\times\{2,4\} = \{(4,2), (4,4), (2,2), (2,4)\}$
\item $\{0\}\times\emptyset = \emptyset$
\item $\{1,2\}\times\{a\} = \{(1,a), (2,a) \}$
\item $\{\emptyset\}\times\{a\} = \{(\emptyset, a)\}$ \\
\textbf{Redo note:} Defined pair \\  
\end{enumerate}

\item Suppose $R \define 2 \upto 5$ and $S == 4 \upto 6$. Enumerate the elements of the following sets.

\begin{enumerate}[a.]
\item $R \cup S = \{2,3,4,5,6\}$
\item $R \cap S = \{4, 5\}$
\item $R \setminus S = \{2,3\}$
\item $S \setminus R = \{6\}$
\item $S \cross R = \{(4,2), (4,3), (4,4), (4,5), (5,2), (5,3), (5,4), (5,5), (6,2), (6,3), (6,4), (6,5)\}$
\end{enumerate}

\item Let $S$ be the set of numbers from 1 to 12 inclusive. Let $R$ be a relation, such that
$R:S\leftrightarrow S$ and such that $x$ is related to $y$ exactly when
$y$ is greater than the square of $x$ but less than the square of
$x+1$. Provide an axiomatic definition for $R$. \\
(Note: be sure to check your notation and formatting --- refer to page 152 in GWC10.)

\item Suppose $\mathit{Let}$ and $\mathit{Num}$ are defined as
follows:
\begin{eqnarray*}
\mathit{Let} & \define & \{a,b,c,d,e\}\\
\mathit{Num} & \define & \{1,2,3,4,5\}
\end{eqnarray*}
\begin{enumerate}[a.]
\item Give an example of each of the following:
\begin{enumerate}[i.]
\item A function whose declaration is $\mathit{Let\rightarrow Num} \\f1 == \{ a \mapsto 1, b \mapsto 2, c \mapsto 3, d \mapsto 4, e \mapsto 5 \}$
\item A function whose declaration is $\mathit{Let\rightarrow\!\!\!\!\shortmid ~Num} \\f2 == \{ a \mapsto 1, b \mapsto 2, c \mapsto 3\}$
\item A total injection from \emph{Let} to \emph{Num}\\
$f3 == \{ a \mapsto 1, b \mapsto 2, c \mapsto 3, d \mapsto 4, e \mapsto 5 \}$\\  
\end{enumerate}
\item Is it possible to give an example of a total injection from \emph{Let} to $\{1,2,3,4\}$? If so, provide one; if not, explain why not. \\
No because the number of values in the source is larger that the target and we can not map more than one source element to the same target because that will violate the injection rule. 

\end{enumerate}
\end{enumerate}
\clearpage

\item \textbf{Proof: Equational Reasoning} \\[8pt]
You may use any of the following theorems in your equational proofs:

\begin{tabular}{llll}
$\vdash p \land \mathit{true} ~~\iff~~ p$   &$\land$-True\\
$\vdash p \land \mathit{false} ~~\iff~~ \mathit{false}$
&$\land$-False\\
$\vdash p \lor \mathit{true} ~~\iff~~ \mathit{true}$ &$\lor$-True\\
$\vdash p \lor \mathit{false} ~~\iff~~ p$ & $\lor$-False\\
$\vdash p \lor \neg p$  &Excluded Middle\\
$\vdash p \lor q ~~\iff~~ q \lor p$ & $\lor$-Commutativity\\
$\vdash p \land q ~~\iff~~ q \land p$ &$\land$-Commutativity\\
$\vdash (p \lor q) \lor r ~~\iff~~ p \lor (q \lor r)$ &$\lor$-Associativity\\
$\vdash (p \land q) \land r ~~\iff~~ p \land (q \land r)$
&$\land$-Associativity\\
$\vdash p \lor (q \land r) ~~\iff~~ (p \lor q) \land (p \lor r)$
&$\lor\land$-Distributivity\\
$\vdash p \land (q \lor r) ~~\iff~~ (p \land q) \lor (p \land r)$
&$\land\lor$-Distributivity\\
$\vdash p \implies q ~~\iff~~ \neg p \lor q$ &$\implies$-Alternative\\
$\vdash p \implies q ~~\iff~~ \neg q \implies \neg p$
&Contrapositives\\
$\vdash \neg\neg p ~~\iff~~ p$   &Double Negation\\
$\vdash \neg(p \land q) ~~\iff~~ \neg p \lor \neg q$&De Morgan\\
$\vdash \neg(p \lor q) ~~\iff~~ \neg p \land \neg q$& De Morgan\\
$\vdash x\in\emptyset ~~\iff~~ \mathit{false}$& $\emptyset$ Membership\\
\end{tabular}
\begin{enumerate}[1.]\setcounter{enumii}{6}
\item Prove in equational style the following laws for set union:

\begin{enumerate}[a.]
\item $S\cup T=T\cup S$ \\
  \textbf{Redo note:} Adjusted proof \\
 \begin{tabular}{l ll lll}
     $x \in S \cup T $  &    &  \\
     $\iff$ &   & [Definition of $\cup$] \\
     $x \in S \lor x \in T$  &    &  \\
     $\iff$ &   & [$\lor$ commutative] \\
     $ x \in T \lor x \in S$  & & \\
     $\iff$  &   & [Definition of $\cup$]  \\
     $x \in T \cup S $  &    &  \\
 \end{tabular} \\
 $S\cup T=T\cup S$ QED \\
\\  
\item $S\cap\emptyset= \emptyset$ \\
  \textbf{Redo note:} Adjusted proof \\
  \begin{tabular}{l ll lll}
     $ x \in S\cap\emptyset$  &    &  \\
    $\iff$ &   & [Definition of $\cap$] \\
    $x \in S \land x \in \emptyset $ & & \\
    $\iff$ &   & [$emptyset$ Membership]\\
    $ x \in T \land false $ & &\\
    $\iff$ &   & [arithmetic] \\
    ${\emptyset}$ & & \\
  \end{tabular} \\
  $S\cap\emptyset= \emptyset$ QED \\
\\ 
\end{enumerate}
(\textsc{Hint}: To prove $S=T$ show $\forall x: U \bullet x \in S\Leftrightarrow x\in T$, where U is the type of elements in sets $S$ and $T$.)
\item Prove the following theorem in equational style:
$$ \vdash \neg(\neg p \implies (q\land r))~~\iff~~(\neg p\land \neg q)\lor(\neg p\land\neg r)$$

\end{enumerate}

\item \textbf{Sequences}


\begin{enumerate}[1.]\setcounter{enumii}{8}

\item Define the following sequences by enumeration:
\begin{enumerate}[a.]
\item $Threes$: natural numbers smaller than 18 that are divisible by 3. \\
  Using GWC10 notation for seq enums definition: \\
  $Threes== <3 , 6 ,  9 , 12 , 15>$ \\
\item  $Twos$: natural numbers smaller than 20 that are divisible by 2. \\
  Using GWC10 notation for seq enums definition: \\
  $Twos== <2 , 4 , 6 , 8 , 10 , 12 , 14 , 16 , 18>$ \\
\end{enumerate}

\item Given the above, what are each of the following:
\begin{enumerate}[a.]
\item $Threes \cup Twos$ \\
  \textbf{Redo note:} Transformed to sets and applied $\cup$. \\ 
  $TreesAsSetOfPairs = \{(1,3),(2,6),(3,9),(4,12),(5,15) \} $ \\
  $TwosAsSetOfPairs = \{ (1,2),(2,4),(3,6), (4,8), (5,10),(6,12),(7,14), (8,16), (9,18) \}$ \\
  $Threes \cup Twos$ = $\{(1,3),(2,6),(3,9),(4,12),(5,15),(1,2),(2,4),(3,6),(4,8),(5,10),(6,12),(7,14),(8,16),(9,18) \}$ \\
\item $Threes \cap Twos$ \\
  \textbf{Redo note:} Transformed to sets and applied $\cap$. \\
  $Threes \cap Twos$ = $\{\emptyset\}$ \\
\item $\dom Threes == \{1, 2 ,3 ,4 ,5 \}$ \\
  To get the domain of a sequence, we just need to get the indexes from 1 up to the cardinality of the sequence. \\
\item $\ran Threes == \{3 ,6 ,9 ,12 ,15\}$ \\
\item $\dom Twos \dres Threes$ \\
  $\dres$ is the domain restrictor operator, informally, the restricted relation contains those \\
  elements from Three whose first component appears in Two. \\
  $dom Twos == \{1,2,3,4,5,6,7,8,9,10,11,12\}$\\
  \textbf{Redo note:} Corrected set notation. \\
  $\dom Twos \dres Threes ==<3, 6, 9, 12, 15>$\\
\item $(5 \upto 8)\dres (Threes \frown Twos)$\\
  $Threes \frown Twos == <3,6,9,12,15,2,4,6,8,10,12,14,16,18>$ \\
  $\dres$ is asking us to filter elements from 5 $\upto 8$ \\
  \textbf{Redo note:} Corrected notation to set and swaped 5 to 6. \\
  $(5 \upto 8)\dres (Threes \frown Twos) == \{15,2,4,6\}$
(\textsc{Note}: $\frown$ is the concatenation operator).
\end{enumerate}
\end{enumerate}


\item \textbf{Proof: Natural Induction} \\[8pt]
Prove the following claim by induction over the natural numbers:
\[\displaystyle 0^2 + 1^2 + 2^2 + ... + n^2 = \frac{n( n + 1 )( 2n + 1 )}{6}\] \\
\textbf{Base case:}\\
$P(0) = \frac{0} {6} = 0 $ \\
$0^2 = 0$ \\
\textbf{Inductive step} We assume that $k \in \mathbb{N}$ and P(k) holds. We then show that P(k+1) holds, that is $0^2 + 1^2 + 2^2 + ... + k + (k + 1)^2 = \frac{(k+1)( (k+1) + 1 )( 2(k+1) + 1 )}{6} = \frac{2k^3 + 9k^2 + 13K + 6}{6}$
\\
\begin{tabular}{l ll lll}
     $0^2 + 1^2 + 2^2 + ... + k + (k + 1)^2$  &    &  \\
     = &   & [substitution, induction hypothesis] \\
     $\frac{k( k + 1 )( 2k + 1 )}{6} + (k + 1)^2$  &    &  \\
     = &   & [arithmetic] \\
     $\frac{(k^2 + k)(2k + 1)}{6} + (k + 1)^2 $ & & \\
     = &   & [arithmetic] \\
     $\frac{2k^3 + 9k^2 + 13K + 6}{6}$  & & \\
\end{tabular} \\

\item \textbf{Proof: Structural Induction} \\[8pt]
Consider the definition of binary trees in Chapter 7.
\begin{enumerate}
\item Show that
  \[\forall t: TREE \bullet leaves(t) = nodes(t) + 1\] \\
\textbf{Redo Note:} Added proof: \\
$ size: TREE \rightarrow \mathbb{N} $\\
------------------------------ \\
$ \forall t1,t2: TREE \bullet $\\
$ size (leaf) = 1 \land $\\
$ size (node(t1,t2) = 1 + size(t1) + size(t2) $ \\
\\
$leaves : TREE \rightarrow \mathbb{N}$ \\
$nodes: TREE \rightarrow \mathbb{N}$ \\
----------------------------- \\
$ \forall t1,t2: TREE \bullet $\\
$ leaves (leaf) = 1 \land $\\
$ leaves (node(t1,t2)) = leaves(t1) + leaves(t2) \land  $ \\
$ nodes(leaf) = 0 \land$\\
$ nodes(node(t1,t2)) = 1 + nodes(t1) + nodes(t2)$ \\
\\
\textbf{Base Case:} Show the property holds for leaf, that is, leaves(leaf) = 1 \\
\begin{tabular}{l ll lll}
  $leaves(leaf)$  &  &  \\
  =  & & [definition of leaves] \\
  0 + 1 & & \\
  = & & [definiton of nodes] \\
  nodes(leaf) + 1 & & \\
  
\end{tabular} \\
\\
\textbf{Induction case:} Generalize base case for leaves(t1) = nodes(t1) + 1 and leaves(t2) = nodes(t2) + 1  \\
\\
\begin{tabular}{l ll lll}
  leaves(node(t1,t2)) &  &  \\
  = & & [applying definiton of leaves] \\
  leaves(t1) + leaves(t2) & & \\
  = & & [induction hypotesis] \\
  nodes(t1) + 1 + nodes(t2) + 1 \\
  = & & [transforming the nodes back to leaves using the nodes(node(t1,t2)) equation]
  nodes(t) + 1 & & \\
\end{tabular} \\
\\


\item Define a $mirror$ function that recursively swaps the branches of a tree. \\
  The idea of this function is that it recursively calls mirror swapping t1 and t2 \\
  every time that it finds a node. If it finds a leaf it just returns. \\
  \\
  $ mirror: TREE \rightarrow TREE $\\
  --------------------------- \\
  \textbf{Redo note:} fixed equation, the key aspect is changing the order of t1, t2 and calling $mirror(node)$ recursively. \\
  $\forall t1,t2 : TREE \bullet$\\
  $mirror(leaf) = leaf \land$ \\
  $mirror(node(t1, t2)) = node(mirror(t2), mirror(t1))$ \\
\item Using the definition of $mirror$  show
that
\[\forall t: TREE \bullet size(mirror(t)) = size(t)\] \\
Using the definition of size(t) provided in GWC10: \\
\\
$ size: TREE \rightarrow \mathbb{N} $\\
----------------------------- \\
$ \forall t1,t2: TREE \bullet $\\
$ size (leaf) = 1 \land $\\
$ size (node(t1,t2) = 1 + size(t1) + size(t2) $ \\
\\
\textbf{Proof:}\\
\textbf{Base Case:} Show the property holds for leaf, that is, size(leaf) = size(mirror(leaf)) \\
\textbf{Redo note:} Corrected equation for mirror(node). \\
\begin{tabular}{l ll lll}
     size(leaf) &    &  \\
     = &   & [definition size(mirror(leaf))] \\
     size(mirror(leaf)) & &  \\
     = &  &  [since mirror(leaf) = leaf]
     size(leaf) &    &  \\
     1  &   & \\
\end{tabular} \\
\\
\textbf{Redo note:} Corrected induction hypotesis. \\
\textbf{Induction case:} Assume that the property holds for trees t1 and t2, that is size(mirror(t1)) = size(t1), and size(mirror(t2)) = size(t2). Show that it holds for node(t1,t2) \\
\\
\begin{tabular}{l ll lll}
     size(mirror(node(t1,t2))) &    &  \\
     = &   & [definition of mirror(node)] \\
     size(node(mirror(t2), mirror(t1)))  & & \\
     = &   & [applying distributive property of size]
     size(mirror(t2)) + size(mirror(t1)) = &   & [induction hypothesis] \\
     size(t2) + size(t1) & & \\
     = &   & [applying distributive property of size] \\
     size(t) & & \\
\end{tabular} \\
\\

\item Using the definition of $mirror$ show that
\[\forall t: TREE \bullet mirror(mirror(t)) = t\]
\end{enumerate}
(\textsc{Hint}: use structural induction over trees.)
\\
\textbf{Proof:}\\
\textbf{Base Case:} Show the property holds for leaf, that is, leaf = mirror(leaf) \\
\\
\begin{tabular}{l ll lll}
     mirror(mirror(leaf)) &    &  \\
     = &   & [applying definition mirror(leaf)] \\
     mirror(leaf) & &  \\
     =   &    & [applying definition of mirror(leaf) \\
     leaf QED \\
\end{tabular} \\
\\
\textbf{Induction case:} Assume that the property holds for trees t1 and t2, that is mirror(mirror(node(t1,t2))) = node(t1,t2) \\
\\
\begin{tabular}{l ll lll}
     mirror(mirror(node(t1,t2)))  &    &  \\
     = &   & [definition of mirror] \\
     mirror(node(mirror(t2), mirror(t1))) & & \\
     = &   & [applying mirror function again] \\
     node(node(t1',t2'), node(node(t2',t1')) & & \\
     = &  & [if t1 = (node(t1',t2') and t2 = node(t2', t1')] \\
    node(t1, t2) QED
\end{tabular} \\
\\

\end{enumerate}

\end{document}
